\chapter{통신이란 무엇인가?}
통신이란 무엇일까? 통신은 어떤 과정을 거쳐서 이루어지는 것일까?
공부를 하다가 종종 길을 잃은 경험이 있다. 내가 뭘 위해서 이것을 하는지를 모르고 한창 공부하다 보니, 갈팡질팡하게 된 것이다.
모 교수님께선 첫 수업이 가장 중요하다고 하셨다. 아마도 첫 수업에서 우리가 무엇을, 왜 공부하는지 알게 되니 그런 말씀을 하신 것일 듯하다.
본격적으로 통신을 공부하기 전에, 우리가 무엇을 왜 공부하는지 잠깐 생각해 보고 넘어가자.

\section{통신의 정의}
통신이란\index{통신} 간단히 말해서 정보를 주고받는 것이다. 그럼 정보\index{정보}\label{정보}는 무엇일까?뒤에서 살펴보겠지만, 정보란 내가 몰랐던, 새로우면서 유용한 내용 정도로 생각할 수 있겠다.
어떤 주식이 언제 오르고 내릴지를 증권사 애널리스트가 내게 알려준다면 얼마나 좋을까?
이 때 애널리스트가 알려주는 것이 바로 정보이고, 알려주는 방법이 바로 통신의 예일 것이다. 메일로 알려줄 수도 있고, 직접 만나서 얘기를 들을 수도 있을 것이다. 모두 통신이다.
\par
통신을 잘 하는 방법은 그렇다면 무엇일까?
이 질문 전에, 통신을 잘 하는 것이 무엇인지부터 생각해보자.
통신이 빨리 될수록, 정보를 정확히 전달할수록, 경제적일수록 통신을 잘 하는 것이라 보면 충분할 것 같다.
그렇다면 통신을 잘 하는 방법은, 통신 속도를 저해하거나 정확도를 떨어뜨리는 요소들을 이해하고, 효율적으로 이 요소들에 대응하는 방법이라고 정의할 수 있겠다.
이미 독자분들이 들어보셨을 변조란 개념이 필요한 이유가 바로 통신을 잘 하기 위함이다.
그렇다면, 통신에는 어떤 방법들이 있을까?

\section{통신의 분류}
통신을 분류하는 방법은 기준이 무엇이냐에 따라 여러 가지가 될 것이다. 각 방법마다 장단점이 있을 것이며, 어느 하나가 모든 면에서 우월할 수는 없다. 각 방법들의 장단점을 앞으로 공부하게 될 것이다.
몇 가지의 기준에 따른 분류를 살펴보자.
\paragraph{아날로그 vs 디지털} 먼저 아날로그 방식으로 통신하느냐 디지털 방식으로 통신하느냐에 따라 통신을 나눌 수 있겠다.
대충 말하자면 아날로그는 연속적인 값을 갖는 것이고, 디지털은 불연속적인 값을 갖는 것이다.
이 문서에서는 다른 책들이 그렇듯이 크게 이 분류를 따라서 설명을 진행하겠다.
\paragraph{기저대역 vs 대역통과} 통신 신호의 주파수들이 어디를 중심으로 분포하느냐에 따라 나눌 수 있다.
기저대역이란 0(직류) 근처에 주파수들이 모여있다는 것이고, 대역통과는 0보다 훨씬 높은 특정 주파수를 중심으로 통신 신호의 주파수들이 모여있다는 뜻이다.
이 또한 살펴볼 것이다.
\paragraph{선형 변조 vs 비선형 변조} 변조가 선형적인 방식인지, 비선형적인 방식인지에 따라서 나눌 수 있다. 진폭 변조, 각 변조가 각각 선형 변조와 비선형 변조의 예이다.
\paragraph{유선 vs 무선} 정보 신호가 전달되는 매체가 유선인지 무선인지에 따라서도 나눌 수 있겠다.

\section{통신의 단계}
통신이 이루어지는 단계는 간단하다. 전달할 정보가 있으면 그 정보를 송신자가 송신하고, 어떤 매체를 거쳐서 수신자가 수신한다. 이게 끝이다.
우리가 배우는 것들은 여기에 통신을 잘 하기 위한 약간의 것들이 더 들어가는 것뿐이다. \figurename~\ref{fig:comm system}\을 보라. 간단하지 않은가?
\begin{figure}[!hbp]
    \centering
    \image[width=12cm]{Communication-System.png}
    \caption{통신 시스템. 간단하다.}\label{fig:comm system}
\end{figure}