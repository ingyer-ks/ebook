\documentclass{book}

\usepackage[unicode=true,
            pdfauthor={ingyer ks},
            pdftitle={기초통신이론},
            pdfsubject={통신에 대해 다루고 싶어하는 문서},
            pdfkeywords={통신},
            colorlinks=true,
            linkcolor=blue,
            citecolor=blue,
            anchorcolor=blue]{hyperref}

\usepackage[hangul]{kotex}
\usepackage{makeidx}
\usepackage{graphicx}
\graphicspath{{../images/}}

\usepackage{silence}
\WarningsOff*

\usepackage{xcolor}

\usepackage{amsmath,amssymb}
\usepackage{IEEEtrantools}

\usepackage{fontspec}
\hangulfontspec{본명조}
\hangulfontspec{본고딕}
\setmainhangulfont{본명조}
\setsanshangulfont{본고딕}

\defaultfontfeatures{Ligature=TeX}

\raggedbottom

\usepackage{tabu}

\usepackage[framemethod=tikz]{mdframed}
\newmdtheoremenv[%
    topline=false, bottomline=false,rightline=false,
    outerlinewidth=3pt, outerlinecolor=DeepSkyBlue!80,
    backgroundcolor=Cyan!5,
    splittopskip=10pt, splitbottomskip=10pt,
    skipabove=5pt, skipbelow=5pt,
    innertopmargin=\topskip, innerbottommargin=\topskip,
]{Thmbox}{정리}[chapter]
\newmdenv[%
    linecolor=MediumPurple,skipabove=10pt,
    roundcorner=5pt,backgroundcolor=MediumPurple!10,
    ]{Problem}
\newmdenv[%
    linecolor=green!10, skipabove=10pt,
    roundcorner=5pt, backgroundcolor=green!8,
    ]{Solution}
\newmdenv[%
    linecolor=blue!10, skipabove=10pt,
    roundcorner=0pt, backgroundcolor=blue!5,
    ]{Stress}

\hypersetup{pdfencoding=auto}

\newcommand{\image}[2][width=12cm]{\includegraphics[#1]{#2}}

\makeindex

\title{기초 통신 이론}
\author{ingyer ks}
\date{2020년 12월 15일}

\begin{document}

\maketitle
\tableofcontents

\chapter{이 문서에 대해}

\section{문서 작성 경위와 목표}
올해(2020년) 상반기, 늘 그랬듯이 뭘 지를까 고민하던 나는 쓸데없이 NAS가 사고 싶었다.
그래서 로켓처럼 빠른 곳에서 NAS 본체와 하드디스크들을 주문했다.
그러고 나니 지름을 합리화할 무언가가 필요해서 블로그를 시작했다. 처음엔 특별한 목적이 있는 것은 아니었다.
그냥 서버 만들기와 관리하기 연습도 해볼 겸 시작했다.
그러다, 코로나-19가 퍼지기 시작하자 집에 있는 시간이 많아졌다.
심심했던 나는 뭘 할까 고민하다가, 각종 통신직 공무원/군무원/공기업 취업 방에 들어갔다.
거기서 놀다 보니 어느날은 블로그에 전자공학, 그 중에서도 통신 관련 내용과 문제풀이를 써보면 어떨까 하는 생각이 들었다.
그래서 \href{https://blog.ingyerlog.kr}{ingyerlog-잉여롭게} 블로그를 시작하게 되었다.
그러다 한발 더 나아가서 \href{https://youtube.ingyerlog.kr}{유튜브}도 잠깐 하게 되었다.
\par
이런 일들을 하면서 든 생각은, 내가 학교에서 공부할 때 사용한 책들은 너무 방대하고, 수험서들은 설명이 빈약하다는 것이었다.
그 중간 정도에서, 통신직 시험을 위해 공부할 때 도움이 될 법한 책은 없을까 하는 생각이 들었다.
물론 있긴 하겠지만, 언젠가는 내가 한번 뭔가 만들어보면 어떨까 하는 생각도 하게 되었다.
그래서 gitbook을 이용해볼까도 하고, 워드프로세서로 문서 작성 시작을 해볼까 고민도 했었다.(고민만)
그러던 어느 날, 2020년이 거의 다 간 지금 12월 중순 즈음, 이전에 몇 번 들어만 보았던 \LaTeX에 불현듯 꽂히게 되었다.
업무 문서도 \LaTeX로 만들어볼까 싶기도 하였고, 또 내가 생각해오던 걸 하는 데 바로 이것이 적합하지 않으면 무엇이 적합하랴 싶었다.
그래서, 책도 질렀고, 시작하려면 지금 시작해야겠단 생각이 들어서, 오늘-이 아니라 어제, 12시가 어느새 넘었다-이렇게 글을 쓰기 시작했다.
다 쓰는 데에 얼마나 걸리려나...
\par
위에서 밝혔듯이, 학교 전공 과목 교재보다는 가볍게, 문제풀이용 수험서보다는 자세하게 내용을 다뤄보고자 한다.
학교 수업에서도 두꺼운 전공책 전체를 다루는 일은 생각보다 적지 않은가?
그렇다고 수험용 책을 보자니 설명이 부족한 것을 몇 번 보았다.
그 중간 지점에서 편한 마음으로 읽어갈 수 있는 문서를 만들고자 한다.

\section{추천 도서 및 문서}
더 자세하고 정확한 내용을 원한다면 아래 문서와 책들을 참고하면 좋겠다.
\begin{itemize}
    \item Haykin \& Moher. (2007). Haykin의 통신이론 (최형진, 공형윤, 김기도, 김대진, 김철성, 이태홍, 장태규 역). 서울:한티미디어. (원서출판 2006) \\
    - 그리 두껍지 않으면서 확률까지 다루는 알찬 책이다.
    \item Richard Baraniuk. (2015). \href{https://cnx.org/contents/77608400-65b9-4f03-8a5f-536c611866bb@15.4/Signals-and-Systems}{Signals and Systems}. \\
    - 기초 신호와 시스템 장을 쓸 때 많은 도움을 받았다. 거의 번역했다고 봐도 될 것이다. \href{https://creativecommons.org/licenses/by/4.0/deed.ko}{크리에이티브 커먼즈 라이센스 저작자표시 4.0 국제(CC BY 4.0) 라이선스}를 따른다.
\end{itemize}

\section{피드백}
잘못된 내용을 발견하였거나 무엇이든 제안할 거리가 있다면 아래 경로로 알려주세요.
\begin{itemize}
    \item \href{mailto:ingyer.ks@gmail.com}{ingyer.ks@gmail.com}
    \item \href{https://github.com/ingyer-ks/Book}{github 레포지토리} (https://github.com/ingyer-ks/Book)
    \item \href{https://kakaotalk.ingyerlog.kr}{카카오톡 오픈채팅} (https://kakaotalk.ingyerlog.kr)
\end{itemize}
\section{도움을 주신 분들}
이 문서를 만드는 데 도움을 주신 분들이 많이 계신다.
전공책을 보겠냐 너가 쓴 글을 보겠냐라고 현실적인 조언을 해 주신 L모 사무관님(그건 형이 너무 잘나서 그런 거에요!!),
지식재산권과 관련하여 대충 보고 문제 없을 거 같다고 안심시켜준 J모 변리사 형, 좋은 글이 될 것 같다며 이런저런 조언을 해준, 곧 백수 생활 청산할 L형, 
남극에서도 글을 봐주고 계신(다고 믿는) N님, 초기 리뷰어가 되어 주신 공무원/군무원/공기업 오픈카톡방의 眞통신님, 유댕님, 안녕하세용님, 두두님 등이 계신다.
사촌동생 K선생, 엄마, 동생 등은 새로운 눈으로 글을 보아 주었다.
\href{http://www.ktword.co.kr/}{정보통신기술용어해설}의 차재복 선생님도 빼놓을 수 없다. 선생님이 계셔서 블로그도 시작하고, 이 글도 시작할 수 있었다. 늘 건강하셨음 좋겠다.
(정보통신기술용어해설에 내 블로그가 걸려있는 게 민망한 일이 되지 않아야 할 터인데...)
아무튼, 크고작은 도움을 주시는 모든 분들을 지속적으로 여기에 기록해나가고자 한다.

\section{지식재산권 관련}
주로 내가 학교에서 배우며 정리한 노트를 중심으로 내용을 채워갔으니 지식재산권 침해 문제는 없을 것이다.(없길 바란다 ㅠㅠ) 
그림들이나 그래프들은 내가 직접 그리거나, 크리에이티브 커먼즈 라이선스 등 사용 조건에 맞는 것들을 찾아서 사용하였다.
가끔 소개할 문제들은 \href{https://www.gosi.kr}{인사혁신처 사이버국가고시센터}와 \href{http://gosi.seoul.go.kr}{서울시인터넷원서접수센터}에 공개되어 있는 문제들이다.
인사혁신처 담당자로부터 비상업적 목적으로 사용하는 것은 문제가 없음을 구두로 확인받았다.
이 문서의 내용 중 내게 권리가 있는 콘텐츠에는 \href{https://creativecommons.org/licenses/by-nc/4.0/}{크리에이티브 커먼즈 저작자표시-비영리 4.0 국제 라이선스(CC BY-NC 4.0)}가 적용된다.
\\
\includegraphics{CC.png}
\chapter{통신이란 무엇인가?}
통신이란 무엇일까? 통신은 어떤 과정을 거쳐서 이루어지는 것일까?
공부를 하다가 종종 길을 잃은 경험이 있다. 내가 뭘 위해서 이것을 하는지를 모르고 한창 공부하다 보니, 갈팡질팡하게 된 것이다.
모 교수님께선 첫 수업이 가장 중요하다고 하셨다. 아마도 첫 수업에서 우리가 무엇을, 왜 공부하는지 알게 되니 그런 말씀을 하신 것일 듯하다.
본격적으로 통신을 공부하기 전에, 우리가 무엇을 왜 공부하는지 잠깐 생각해 보고 넘어가자.

\section{통신의 정의}
통신이란\index{통신} 간단히 말해서 정보를 주고받는 것이다. 그럼 정보\index{정보}\label{정보}는 무엇일까?뒤에서 살펴보겠지만, 정보란 내가 몰랐던, 새로우면서 유용한 내용 정도로 생각할 수 있겠다.
어떤 주식이 언제 오르고 내릴지를 증권사 애널리스트가 내게 알려준다면 얼마나 좋을까?
이 때 애널리스트가 알려주는 것이 바로 정보이고, 알려주는 방법이 바로 통신의 예일 것이다. 메일로 알려줄 수도 있고, 직접 만나서 얘기를 들을 수도 있을 것이다. 모두 통신이다.
\par
통신을 잘 하는 방법은 그렇다면 무엇일까?
이 질문 전에, 통신을 잘 하는 것이 무엇인지부터 생각해보자.
통신이 빨리 될수록, 정보를 정확히 전달할수록, 경제적일수록 통신을 잘 하는 것이라 보면 충분할 것 같다.
그렇다면 통신을 잘 하는 방법은, 통신 속도를 저해하거나 정확도를 떨어뜨리는 요소들을 이해하고, 효율적으로 이 요소들에 대응하는 방법이라고 정의할 수 있겠다.
이미 독자분들이 들어보셨을 변조란 개념이 필요한 이유가 바로 통신을 잘 하기 위함이다.
그렇다면, 통신에는 어떤 방법들이 있을까?

\section{통신의 분류}
통신을 분류하는 방법은 기준이 무엇이냐에 따라 여러 가지가 될 것이다. 각 방법마다 장단점이 있을 것이며, 어느 하나가 모든 면에서 우월할 수는 없다. 각 방법들의 장단점을 앞으로 공부하게 될 것이다.
몇 가지의 기준에 따른 분류를 살펴보자.
\paragraph{아날로그 vs 디지털} 먼저 아날로그 방식으로 통신하느냐 디지털 방식으로 통신하느냐에 따라 통신을 나눌 수 있겠다.
대충 말하자면 아날로그는 연속적인 값을 갖는 것이고, 디지털은 불연속적인 값을 갖는 것이다.
이 문서에서는 다른 책들이 그렇듯이 크게 이 분류를 따라서 설명을 진행하겠다.
\paragraph{기저대역 vs 대역통과} 통신 신호의 주파수들이 어디를 중심으로 분포하느냐에 따라 나눌 수 있다.
기저대역이란 0(직류) 근처에 주파수들이 모여있다는 것이고, 대역통과는 0보다 훨씬 높은 특정 주파수를 중심으로 통신 신호의 주파수들이 모여있다는 뜻이다.
이 또한 살펴볼 것이다.
\paragraph{선형 변조 vs 비선형 변조} 변조가 선형적인 방식인지, 비선형적인 방식인지에 따라서 나눌 수 있다. 진폭 변조, 각 변조가 각각 선형 변조와 비선형 변조의 예이다.
\paragraph{유선 vs 무선} 정보 신호가 전달되는 매체가 유선인지 무선인지에 따라서도 나눌 수 있겠다.

\section{통신의 단계}
통신이 이루어지는 단계는 간단하다. 전달할 정보가 있으면 그 정보를 송신자가 송신하고, 어떤 매체를 거쳐서 수신자가 수신한다. 이게 끝이다.
우리가 배우는 것들은 여기에 통신을 잘 하기 위한 약간의 것들이 더 들어가는 것뿐이다. \figurename~\ref{fig:comm system}\을 보라. 간단하지 않은가?
\begin{figure}[!hbp]
    \centering
    \image[width=12cm]{Communication-System.png}
    \caption{통신 시스템. 간단하다.}\label{fig:comm system}
\end{figure}
\chapter{기초 신호와 시스템}
통신은 정보를 주고받는 것이라고 정의했다. 그런데, 어떤 형태로 정보를 주고받을까?
바로 전기 신호이다. 그리고 통신을 하는 장치들은 각각 하나의 시스템으로 볼 수 있다.
따라서, 우리는 신호에 대해서 공부하고, 시스템이 신호를 어떻게 다루는지 알 필요가 있다.
본격적으로 통신에 대해 배우기 전에 신호와 시스템에 대해서 공부하자.

\section{신호}
\subsection{신호의 정의와 신호의 해석 관점}
신호란 무엇일까? 사전을 찾아보니 추상적으로는 어떤 내용이 담겨 있는 행동, 부호 등이라고 한다.
봉화를 올려 적의 침입을 알리는 것, 초록색이 점등되면 차가 갈 수 있음을 알려주는 신호등, 이런 것들이 다 신호를 사용하는 것이다.
이처럼 다양한 신호들이 있지만 여기서는 전기 신호, 즉 어떠한 전압 값이나 전류값을 가지는 것을 신호\index{신호}라고 정의하자.
좀 더 구체적으로는 우리는 `전압'을 다룬다고 생각하자. 전류가 아닌 전압을 다루는 이유는 한번 찾아보자. 나도 잘 모르겠으니까.
\par
크게 중요하진 않지만, 보통 간과하고 넘어가는 것을 하나 얘기하고 싶다. 앞으로 우리는 암묵적으로 저항이 $1 \Omega$ 라고 생각할 것이다.
보통 RF 시스템은 $50 \Omega$, $75 \Omega$ 등을 사용하나, 편의를 위해서 $1 \Omega$ 이란 값을 쓰는 것이다.
이렇게 하면 전력은 그냥 전압(신호)를 제곱하는 것이랑만 관련되고, 실제 전력은 실제 저항값으로 나눠주면 될 것이다.
(왜 그냥 제곱하면 된다고 하지 않았을까?)
\par
한편, 신호란 건 변해야 의미가 있다.
변하지 않는 것에 대해 생각해보자.
아름다운 사진, 그림은 물론 가치있는 것이긴 하지만, 우리 기억력이 무한히 좋다면 한번 찬찬히 그림을 보고 나면 더이상 거기서 얻어낼 정보는 없을 것이다.
\pageref{정보}페이지에서 우리가 정보를 어떻게 정의했는지 기억하는가? 새로움이 정보의 특징이었다. 그리고 새롭다는 것은 이전에 비해 변화가 있다는 뜻이다.
`이전'이라는 표현을 잘 보자. 결국 변화란 (적어도 이 문서에서는) 시간이 지남에 따라 달라지는 것을 의미한다.
따라서 우리는 시간에 따라 변하는 전기, 즉 교류(AC)에 대해서 다루게 될 것이다. 
특히 시간에 따라 신호값이 어떻게 되는지, 즉 신호를 시간에 대한 함수로 생각하는 관점을 시간 영역 해석이라고 하자. `영역'이라는 단어는 `도메인'이라고도 한다. 앞으로 이 둘을 혼용하겠다.
\par
다른 관점에서 신호를 바라볼 수는 없을까? 피아노 소리를 생각해보자. 여러 건반이 눌린 소리가 나고 있다. 시간에 따라 기록을 해 보면 여러 건반 각각의 소리들의 합이 기록될 것이다.
그런데, 이 소리들이 얼마나 오랜 시간동안 지속되는지 말고 각 건반들의 소리 높낮이에 관심을 가져보면 어떨까?
즉, 소리의 성분에 관심을 가지자는 것이다. 소리의 높낮이를 결정하는 것은 바로 주파수이다.
어떤 소리가 어떤 주파수들로 이루어졌는지 알아보는 것, 이것이 바로 푸리에 해석\index{푸리에 해석} 또는 주파수 영역 해석\index{주파수 영역 해석}\label{freq domain analysis}이다.
다시 말해, 푸리에 해석은 신호를 주파수 영역에서 해석하는 것이다.
분명 장점이 있으니 주파수 영역 해석이란 것을 하는 것일 게다. 조금 후 시간 영역 해석에 대해 알아본 후 주파수 영역 해석에 대해 알아보자.

\subsection{신호의 분류}
통신을 분류했던 것처럼 신호도 분류를 해보고 넘어가자. 신호의 종류에 따라 적용되는 해석 방법이 달라질 것이다.
\paragraph{연속시간신호 vs 이산시간신호}
신호는 연속적인(continuous) 시간에 대한 함수일 수도 있고 이산적인(discrete) 시간에 대한 함수일 수도 있다.
자동차를 운전할 때, 자동차의 위치를 시간에 대해 나타내보면 연속함수로 나타날 것이다. 반대로, 자동차의 위치를 매 5초마다 측정한다면 자동차의 위치는 시간에 대해 띄엄띄엄 나타나게 된다. \figurename~\ref{fig:cont vs disc}\을 보라.
\begin{figure}[!tbp]
    \centering
    \image{signals_by_time.pdf}
    \caption{연속시간신호와 이산시간신호의 예}\label{fig:cont vs disc}
\end{figure}
\par
앞으로 어떤 신호가 연속 시간 신호라면 $f(t)$와 같이 소괄호를 사용하고, 이산 시간 신호라면 $g[n]$과 같이 대괄호를 사용하겠다.
\paragraph{아날로그 신호 vs 디지털 신호}
연속신호와 이산신호가 시간 축에 대한 것이었다면, 아날로그냐 디지털이냐는 신호가 가질 수 있는 값에 대한 것이다. 아날로그는 연속적인 값을 가지며, 오실로스코프로 찍어봤을 때 나타나는 그림에 대응된다. 즉 물리적인 값이다.
반대로 디지털 신호는 특정 값들만 가질 수 있다. \figurename~\ref{fig:analog vs digital}\을 보라. 
\begin{figure}[!hbp]
    \centering
    \image{analog_and_digital_signals.pdf}
    \caption{아날로그 신호와 디지털 신호의 예}\label{fig:analog vs digital}
\end{figure}
\\
왼쪽의 아날로그 신호는 다양한 값을 가질 수 있지만, 오른쪽의 디지털 신호는 특정 값만 가질 수 있기에 선이 각져 있는 것을 볼 수 있다.
\par
또한 디지털은 추상적인 개념이다. 디지털에서 1이란 값은 물리적으로는 5 V에 해당할 수도 있고, 3.3 V에 해당할 수도 있는 것이다.
이 추상적인 디지털 신호를 실제로 사용하기 위해서는 반드시 물리적인 값(아날로그)로 실체화되어야 한다.
논리적인 0과 1을 어떻게 전달하고 표현할 것인가? 텔레파시를 보낼 것도 아니고 5 V든 3.3 V든 물리적인 값에 대응을 시켜야 물리적인 회로 등을 이용해서 정보를 처리하고 전달할 수 있지 않겠는가?
결국, 어떤 물리적인 실체를 해석하는 것은 사람이 하는 것이다. 상황에 따라 적절하게 아날로그 신호나 디지털 신호로 해석해야 하는 것이다.
\paragraph{주기신호 vs 비주기신호}
특정한 시간이 지난 뒤에 앞서 나온 신호가 다시 나타나는 신호는 주기 신호이고, 이러한 규칙성이 없으면 비주기 신호이다.
주기 신호에 대해서 조금 생각해보자. 만약 $T$라는 시간이 지날 때마다 앞서 나타난 신호가 반복된다고 해보자. 즉 $f(t)=f(t+T)$라고 하자.
이 $f$란 신호는 $T$라는 시간에 대해서만 주기적일까? $T$만큼 시간을 더 진행시켜보자. 그러면 $f(t)=f(t+T)=f(t+T+T)=f(t+2T)$\label{eqn:periodic signal}가 된다. 즉 $2T$도 이 신호의 주기 중 하나가 되는 것이다.
이러한 과정을 반복해보면 결국 주기는 $T$, $2T$, $3T$ 등 정수 $n$에 대해서 모든 $nT$가 주기\index{주기}가 된다. 그리고 이 주기들 중 가장 작은 $T$를 기본 주기\index{기본 주기}라고 한다.
(주의: 고등학교 수학에서 배우는 주기의 정의는 이와 다르다. 고등학교에서는 이 문서에서의 기본 주기를 주기라고 정의한다.)
\begin{figure}[!hbp]
    \centering
    \image{nonperiodic_and_periodic_signals.pdf}
    \caption{비주기 신호와 주기 신호의 예}
\end{figure}
\paragraph{전력신호 vs 에너지신호}
전력과 에너지의 차이점은 무엇일까? 전력은 단위시간 당 에너지를 말하고, 에너지는 그냥 에너지 총량을 말한다.
이로부터 전력신호와 에너지신호가 무엇을 뜻하는지 생각해낼 수 있다.
\par
먼저 전력신호부터 살펴보자. 전력신호는 평균전력이 0이 아니면서 유한한 신호를 말한다. 수식으로 보자.
\begin{equation*}
    0 < P = \lim_{T \rightarrow \infty} \frac{1}{T} \int_{-T/2}^{T/2} {\left\vert x(t) \right\vert}^2 dt < \infty
\end{equation*}
고등학교 때 배운 극한을 생각해 보자. 
\begin{equation*}
    \lim_{x \rightarrow 0} \frac{f(x)}{g(x)} = C (-\infty < C < \infty\text{ and }C \neq 0)
\end{equation*} 라면, $f(x)$와 $g(x)$는 둘 다 수렴하거나, 무한대로 발산하지 않던가? 위 경우에는 $T$가 무한대로 발산하니, $ {\left\vert x(t) \right\vert}^2 $ 또한 무한대로 발산한다.
그런데, 이 값은 바로 $x(t)$의 에너지가 아닌가? 즉 전력 신호의 에너지는 무한대이다.
여기서 왜 이런 신호들을 `전력'신호라고 부르는지 알 수 있다. 어차피 이 분류에 해당하는 신호들은 전부 에너지가 무한대이니 에너지로서는 이들 신호들을 특징지을 수가 없는 것이다.
전력 신호의 예로는 당연히 많은 주기 신호들이 해당할 것이다. 그렇다고 모든 주기 신호가 전력 신호이고, 반대로 모든 전력 신호가 주기 신호라 할 수 있을까? 생각해보라.
\par
다음으로, 에너지 신호에 대해 알아보자. 에너지 신호는 에너지가 0이 아니면서 유한한 신호를 말한다. 수식으로 써보자.
\begin{equation*}
    0 < E = \int_{-\infty}^{\infty} {\left\vert x(t) \right\vert}^2 < \infty
\end{equation*}
이런 신호의 전력을 전력신호 구하듯이 구해보면 얼마일까? $T$는 무한대로 발산하는데 에너지는 한정되어 있으니 전력은 결국 0이 된다.
따라서 이 분류에 속하는 신호들의 전력은 모두 0이므로 전력은 특징적인 값이 될 수 없고, 에너지가 특징이 되는 것이다.
유한한 값을 가지면서 유한한 시간동안만 존재하는 신호들은 모두 에너지 신호일 것이다. 그렇지 않으면서 에너지 신호인 것도 있을까?생각해보자.
\par
한편, 이러한 질문들을 종종 받는다. 램프 신호는 에너지 신호인가 전력 신호인가? 지수부의 부호가 양인 지수 신호는 어디에 속하는가?
정의를 잘 안다면 헷갈릴 게 없다. 정의에 입각하여 판단해보시라. 이로부터 다음 질문에 대한 대답을 할 수 있다: 신호는 모두 전력 신호나 에너지 신호로 분류될 수 있는가?
\begin{figure}[!hbp]
    \centering
    \image{ramp_and_exp.pdf}
    \caption{램프 신호와 지수 신호}
\end{figure}
\paragraph{랜덤신호 vs 결정신호}
우리가 다루는 신호는 주사위를 던졌을 때 나오는 값처럼 확률적인 랜덤신호일 수도 있고, 피아노 소리처럼 미리 약속된 신호일 수도 있다.
랜덤신호라면 당연히 확률론의 개념들(기댓값 등)이 적용되어야 할 것이다.

\subsection{중요한 신호들}
통신 분야에서 중요한 신호들이 몇 개 있다. 살펴보고 넘어가자.
\paragraph{사인파}\index{사인파}\label{사인파} 
사인파는 Sinusoids\index{Sinusoid}라고도 불리는 신호이며, 이미 잘 알겠지만 
\begin{IEEEeqnarray}{rCl}
    y(t)&=&A sin(2\pi ft+\phi )=A sin(\omega t+\phi )\\
    &=&A cos(\omega t+ \phi + \pi/2)\label{eqn:sinusoid representation by cosine}
\end{IEEEeqnarray}
형태의 신호이다.
$A$는 진폭, $f$는 주파수, $\omega=2\pi f$는 각주파수, $\phi$는 위상이다. 이름은 사인파지만 편의성을 이유로 식 \ref{eqn:sinusoid representation by cosine}의 코사인 형태로 많이 사용한다.
이 수식에서 표현되지 않았지만, 주기 $T$는 주파수\label{frequency}의 역수 $\frac{1}{f}$이며 각주파수를 이용해 표현하면 $T=\frac{2\pi}{\omega}$이다.
사인파의 중요한 특징은 각각의 사인파는 하나의 주파수만으로 이루어진다는 것이다. 뒤에서 살펴보겠지만 이를 이용해서 신호들을 여러 사인파들의 합으로 표현할 수 있다.
\begin{figure}[!hbp]
    \centering
    \image{sine_wave.pdf}
    \caption{사인파}
\end{figure}

\paragraph{단위 계단 함수} 단위 계단 함수는 말 그대로 계단처럼 생긴 함수이다. 이 함수를 연구한 영국의 전기공학자 올리버 헤비사이드로부터 따와서 헤비사이드 계단 함수라고도 부른다. 
먼저 \figurename~\ref{fig:usf}\을 보자.
\begin{figure}
    \centering
    \image{unit_step_function.pdf}
    \caption{단위 계단 함수}\label{fig:usf}
\end{figure}
정말 계단처럼 생기지 않았는가? 그리고 높이가 1이기 때문에 `단위' 계단 함수라고 부른다. 수식으로는 다음과 같이 표기한다.
\begin{equation}
    u(t)=\begin{cases}
        0 & \text{if } t < 0,\\
        \frac{1}{2} & \text{if } t=0,\\
        1 & \text{if } t \geqq  0.
    \end{cases}
\end{equation}
$t=0$일 경우는 애매함이 있긴 하나 여기선 그냥 $1/2$로 대충 생각하고 넘어가겠다.
\par
그건 그렇고, 이 단위 계단 함수가 어떻게 쓰이는지 생각해보자. 대표적인 예가 스위치 역할이다. 
예를 들어, 0\~{}5에서는 사인, 5\~{}10에서는 직선, 10\~{}15에서는 삼차함수인 복잡한 그래프가 있다고 하자.
이 그래프를 수식으로 어떻게 나타낼 것인가? 바로 이 단위 계단 함수를 적절히 각 함수들과 곱한 후 다 더하면 된다.

\paragraph{단위 임펄스 함수} 
단위 임펄스 함수는 $\delta (t)$로 표기한다. 연속 시간 신호일 경우에는 디랙 델타 함수라고도 하며, 이산 시간 신호일 경우에는 크로네커 델타 함수라고도 한다.
앞으로 매우 중요하게 다루어질 함수이니 잘 기억해 두자. 앞으로 델타 함수와 임펄스 함수를 혼용하도록 하겠다.
디랙 델타 함수는 $t\neq0$인 곳에서는 값이 $0$이고, $t=0$이면 무한대의 세기(strength)를 갖는 함수로 정의된다. 크기가 아니라 세기라고 표현하였음에 유의하라.
\figurename~\ref{fig:dirac}\을 보라. 일반적인 함수와 달리 화살표를 이용해서 그리고 있다. 그 옆의 1은 $\delta(t)$의 계수 1을 의미한다.
\begin{figure}[!hbp]
    \centering
    \image{dirac_visualization.pdf}
    \caption{디랙 델타 함수}\label{fig:dirac}
\end{figure}
\par
디랙 델타 함수는 다양한 방법으로 유도하고 수식으로 나타낼 수 있으나, 이 문서에서는 다음 식을 사용하겠다.
\begin{equation}
    \delta (t)=\int_{-\infty}^{\infty}e^{\pm j2\pi ft}df\label{eqn:def of delta}
\end{equation}
\par
디랙 델타 함수의 중요한 특징은, $t=0$을 포함하는 구간에 대해서 적분하면 1이 나온다는 것이다.
즉 다음과 같다.
\begin{equation}
    \int_{a}^{b} \delta (t) dt = \begin{cases}
        1 & \text{if } a < 0 < b,\\
        0 & \text{otherwise.}
    \end{cases}
\end{equation}
이를 이용하면 어떤 함수의 특정 지점에서의 함숫값을 뽑아낼 수 있다. 원하는 지점으로 디랙 델타 함수를 이동시켜서 함수에 곱한 후 그 위치를 포함하는 구간에서 정적분해주면 된다.
왜 그렇게 되는지 계산하여 알아보자.
\begin{equation}
    \int_{-\infty}^{\infty}f(t)\delta(t)dt=\int_{-\infty}^{\infty}f(0)\delta(t)dt=f(0)\int_{-\infty}^{\infty}\delta(t)dt=f(0)\cdot 1 = f(0) \label{eqn:f value calc with delta}
\end{equation}
왜 $f(t)$가 $f(0)$이 되었을까? $t\neq0$인 지점들에선 어차피 $\delta(t)$가 0이니 $f(t)$ 자리에 아무 숫자나 써넣어도 계산 결과는 같다.
하지만 $t=0$이라면 $\delta(0)\neq0$이므로 맘대로 $f(0)$ 값을 바꿀 수 없으니 그대로 쓴 것이다.
위 식 \ref{eqn:f value calc with delta}은 매우 중요한 결과이니 잘 기억해 두자. 
또한, 델타 함수는 입력값이 0에 대해 대칭이다. 즉 $\delta(-t)=\delta(t)$이다. 이것은 델타 함수의 정의인 식 \ref{eqn:def of delta}에서 바로 알 수 있다.
\par
다른 중요한 성질은 어떤 게 있을까? 델타 함수의 입력에 다른 계수가 붙으면 어떨까? 계산해 보면
\begin{equation*}
    \int_{-\infty}^{\infty}\delta(\alpha t)dt=\begin{cases}
        \int_{-\infty}^{\infty} \frac{1}{\alpha }\delta(\tau )d\tau=\frac{1}{\alpha } & \text{if } a>0,\\
        \int_{\infty}^{-\infty} \frac{1}{\alpha }\delta(\tau )d\tau= \int_{-\infty}^{\infty} \frac{1}{-\alpha }\delta(\tau )d\tau = \frac{1}{-\alpha } & \text{if } a<0.
    \end{cases}
\end{equation*}
이다. 한편 $\alpha <0$일 때 $-\alpha=\left\vert \alpha \right\vert$이므로
\begin{equation}
    \delta(\alpha t)=\frac{\delta(t)}{\left\vert \alpha \right\vert}
\end{equation}
임을 알 수 있다.
\par
디랙 델타 함수와 단위 계단 함수 사이에는 중요한 관계가 있다. 바로 그 둘이 미적분으로 연결된다는 것이다. 다음 수식을 보자.
\begin{equation}
    \frac{du(t)}{dt} =\delta(t) \label{eqn:usf and delta}
\end{equation}
왜 이렇게 되는 것일까? 고등학교 때까지는 부드러운 곡선이 아니면(또 부드러우면 다 되는 것도 아니고) 미분이 불가능하다고 배웠다. 하지만 미분 연산을 좀 더 넓게 정의해서 사용해보자.
단위 계단 함수의 값이 바뀌는 지점에서의 기울기는 \figurename~\ref{fig:usf}에서 볼 수 있듯이 무한대이다. 즉 $u(t)$를 미분하면 $t=0$에서 무한대가 나온다는 것이고, 이는 결국 수식으로 쓰면 
식 \ref{eqn:usf and delta}가 되는 것이다.
\par
크로네커 델타의 경우는 훨씬 쉽다. \figurename~\ref{fig:kronecher}\을 보면 바로 알 것이다.
\begin{figure}[!hbp]
    \centering
    \image{kronecher_visualization.pdf}
    \caption{크로네커 델타 함수}\label{fig:kronecher}
\end{figure}
수식으로 나타내면 다음과 같다.
\begin{equation}
    \delta[k]=\begin{cases}
        1 & \text{if } k = 0,\\
        0 & \text{elsewhere.}
    \end{cases}
\end{equation}

\section{시스템}
\subsection{시스템의 정의}
위 질문에 대해 한번 생각해보자. 시스템이란 단어는 생각해보면 참 추상적인 단어이다. 
컴퓨터의 운영 체제도 영어로 Operating System이고, 선형대수에서도 시스템이란 단어가 나오고, 반도체 공정에서도 systematic offset이란 단어가 나온다.
그 외에도 시스템이란 단어는 참 많이 쓰인다. 얼핏 보면 상관없는 분야들에서 두루 쓰이는 단어로 느껴진다.
\par
그래도 굳이 공통점을 찾아보자면, 사전에서 찾아봐도 알 수 있지만, 어떤 특정한 방법, 체계, 방식이 바로 시스템의 의미인 것 같다.
이 문서에서는 시스템\index{시스템}을 어떠한 규칙에 따라 작동하는 무언가라고 정의하려고 한다.
이 무언가는 물리적으로 구현된 회로일 수도 있고, 추상적으로 어떤 기능을 한다고 가정한 블랙박스일 수도 있다.
특히 우리는 통신을 공부하고 있으므로, 통신 시스템이란 보내고자 하는 정보를 입력받아 규칙에 따라 잘 처리하여 출력해서 통신이 원활하게 되게 하는 집합체 정도라고 생각하자.
\begin{figure}[!hpb]
    \centering
    \image{system.png}
    \caption{시스템과 입출력}\label{fig:system and io}
\end{figure}
\subsection{시스템의 주요 성질}
시스템이 가질 수 있는 성질들 중 중요한 것이 몇 가지 있다. 매우 중요한 성질들이므로 잘 이해하자.
\paragraph{선형성(Linearity)}\index{선형성}
선형성이란 말 그대로 직선과 같은 성질을 따름을 의미한다. 정확히는 $y=f(x)=kx$와 같은 형태의 직선이다. 이 직선의 특징을 분석해보자.
어떠한 입력 $x_1$이 있다고 하자. 이 $x_1$에 $a$가 곱해져서 $ax_1$이 된다면 $f$는 어떤 출력을 내놓는가? 바로 
\begin{equation*}
    f(ax)=k(ax)=(ka)x=(ak)x=a(kx)
\end{equation*}
라는 출력을 내놓는다.
즉 입력이 $a$배 되면 출력도 $a$배가 되는 것이다. 이러한 성질을 균질성(homogeneity)이라고 부른다.
다음으로, 또 다른 입력 $x_2$가 있다고 하자. 두 입력이 합쳐진다면, 즉 $x_1+x_2$라는 입력이 $f$에 가해진다면 출력은 어떻게 되는가?
\begin{equation*}
    f(x_1+x_2)=k(x_1+x_2)=kx_1+kx_2
\end{equation*}
가 된다. 즉 두 입력이 들어오면, 각 입력에 대한 각각의 출력이 합쳐져서 전체 출력이 되는 것이다.
일반화하면 두 개의 입력이 아니라 어떤 정수 $n$개의 입력이 들어오면 각각의 입력에 대한 출력 $n$개가 다 합쳐져서 나온다.
이 성질을 가산성(additivity)이라고 부르며, 중첩(superposition)이라고도 한다. 
이 두 성질, 균질성과 가산성이 있으면, 또 오직 그럴 때에만 선형성이 있다고 한다. 즉 이 두 성질이 선형성의 정의인 것이다.
\par
선형성이 왜 중요할까? 시스템이 선형적이라면, 우리는 어떠한 입력에 대해서 시스템이 어떤 출력을 내놓을지 쉽게 예상할 수 있다.
선형성이 없다면 벌어지는 것이 바로 카오스이다. 실제로 카오스의 특징 중 하나가 비선형성이다. 
약간의 차이가 있어도 그 비선형적인 특징으로 인해 다음 값이 어찌될지 알 수가 없게 되는 것이다. 그래서 우리는 기를 쓰고 선형성을 추구하게 된다. 다루기가 쉬워지니까.
\paragraph{시불변성(Time-invariance)}\index{시불변성}
시불변성이란 시간이 변할지라도 시스템의 체계는 바뀌지 않는다는 성질이다.
이게 무슨 말인지 더 풀어보자면, 시불변성이 있는 시스템에 어떤 입력을 지연시켜서 넣으면 출력도 지연되어서 나온다는 뜻이다.
수식으로 쓰자면 $y(t)=H(x(t))$라는 시스템에서 $T$만큼 시간이 지연되어 입력된다면 출력은 
\begin{equation*}
    H(x(t-T))=y(t-T)
\end{equation*}
가 된다는 것이다.
당연히, 이런 성질이 있으면 아주 다루기가 쉬울 것이다. 입력이 지연되는 시간만 신경쓰면 되니까.
\paragraph{인과성(Causality)}\index{인과성}\label{causality}
인과성이란 시스템의 출력은 이미 들어온 입력들과 지금 당장의 입력에만 의존한다는 성질이다.
당연히, 물리적으로 구현 가능한 시스템은 시간에 대해 인과성이 있어야 한다.
수식으로 보자면 시스템의 임펄스 응답 $h(t)$는 
\begin{equation}
    h(t)=0\text{ for }t<0 \label{eqn:causality condition}
\end{equation}
이어야 한다. 이를 이해하려면 조금 더 공부를 해야 하니 일단 그렇다고 알고 지나가자. 혹시 미리 알아보고 싶다면 \pageref{인과성 해설} 페이지의 \ref{인과성 해설}\을 참고하라.
\paragraph{안정성(Stability)}\index{안정성}
안정성이란 시스템의 출력이 폭주하지 않는다는 성질이다. 
좀 더 구체적으로 말하자면, 최대 크기가 제한된 입력(Bounded Input)에 대해 출력의 크기도 제한된다면(Bounded Output) 이를 BIBO 안정성이라고 한다.
\subsection{선형 시불변 시스템}
선형 시불변 시스템(Linear and Time-Invariant System;LTI System)\index{선형 시불변 시스템}\index{LTI} 은 말 그대로 선형적이며, 시간에 대해 불변인 시스템이다.
제일 다루기 간단한 시스템이며, 앞으로 우리가 공부할 시스템은 전부 LTI 시스템이라고 생각해도 된다.
LTI 시스템의 출력은 시스템의 임펄스 응답과 입력의 콘볼루션이 된다. 뒤에서 콘볼루션에 대해 집중적으로 살펴볼 것이다.

\section{시간 영역 해석}
앞서 우리는 신호가 무엇이고, 신호를 다루는 것이 시스템이라고 알아봤다. 그리고 신호는 시간에 따라 변하는 것이므로 입력에 대한 시스템의 출력이 시간에 따라 어떻게 나타나는지 알아볼 필요가 있다.
이 장에서는 그러한 방법과 의미를 알아볼 것이다.
\subsection{신호와 임펄스 함수}
어떤 연속 시간 신호 $f(t)$는 임펄스 함수들로 표현될 수 있다. \figurename~\ref{fig:sig by impulses}\을 보면서 구체적으로 어떻게 그렇게 되는지 알아보자.
왼쪽 그림과 같이 신호가 있다고 하자. 그리고 중간 그림처럼 여러 임펄스 함수들로 이루어진 임펄스 트레인이 있다고 하자.
이 둘을 곱하면 오른쪽처럼 신호가 임펄스 함수들로 표현되게 된다.
\begin{figure}
    \centering
    \image[height=3cm]{signal_and_impulse.pdf}
    \caption{임펄스 트레인으로 표현된 신호}\label{fig:sig by impulses}
\end{figure}
이 그림을 수학적으로 살펴보자.
식 \ref{eqn:f value calc with delta}에서 보았듯이, 특정 시각 $T_1$에서의 함숫값을 원한다면 $f(t)$에다 $\delta(t-T_1)$을 곱한 후 $T_1$을 포함하는 구간에 대해 적분하면 된다.
만약 임의의 시각들 $T_n\text{(}n\text{은 정수)}$에 대해서 $f(t)\delta(t-T_n)$들을 만들어 두고, 각 $T_n$들 사이의 간격을 아주 좁게 하여 적분하는 구간을 좁히고 이 정적분들을 모두 더하면 어떻게 될까?
어떤 시각 $T_k$에 대해서 $n\neq k$라면 각 적분들은 $0$을 만들 것이고, $n=k$인 정적분은 $f(T_k)$라는 결과를 낼 것이다.
이 $T_k$가 모든 실수가 될 수 있다면, 우리는 $f(t)$를 무수히 많은 임펄스 함수들, 즉 임펄스 트레인을 이용해서 구성해낸 것이다.
수식으로 쓰자면
\begin{equation}
    f(t)=\int_{-\infty}^{\infty}f(\tau)\delta(t-\tau)d\tau \label{eqn:f by impulse train}
\end{equation}
이다. 다음으로 배울 콘볼루션 개념을 미리 끌어 쓰자면, 어떤 함수 $f(t)$는 자기 자신과 임펄스 함수의 콘볼루션이 된다. 이것이 델타 함수가 중요한 하나의 이유이다.
\par
이산 시간 신호에 대해서도 비슷하게 생각할 수 있다. 이산 시간 버전의 수식은 다음과 같다.
\begin{equation}
    g[n]=\sum _{k=-\infty}^{\infty}g[k]\delta[n-k]
\end{equation}

\subsection{콘볼루션}
종이 하나 있다. 그리고 이 종을 치는 일을 임펄스 충격을 가하는 것으로 가정하자.
이 종은 한번 치면(=임펄스를 가하면) 그 세기에 비례하여 소리가 났다가 지수함수적으로 소리가 감쇄하며, 또 치면 앞서 발생하여 감쇄중인 소리에 새롭게 소리가 더해지는 특징이 있다. 그리고 종의 반응은 시간이 지나도 바뀌지 않는다.
이 종을 같은 간격으로 세 번 쳤을 때의 반응은 무엇일까?
먼저 그림으로 생각해 보자.
\begin{figure}
    \centering
    \image[height=3cm]{sum_of_impulse_response.pdf}
    \caption{종을 세 번 쳤을 때 반응}\label{fig:bell response}
\end{figure}
\figurename~\ref{fig:bell response}\을 보면 각각 처음 종을 쳤을 때, 그 후에 종을 쳤을 때, 마지막으로 종을 쳤을 때의 반응이 그려져 있다. 어느 순간의 종의 반응은 그 순간 각 반응들의 합이 될 것이다.
수식으로 표현하자면 처음 종을 친 순간을 $t=0$, $t$인 순간에 종을 치는 세기를 $f(t)$, 한번 칠 때 종의 반응을 $h(t)$라고 하고, 종을 치는 간격을 $T$라고 하면 종의 총 반응 $y(t)$는 다음과 같게 된다.
\begin{equation}
    y(t)=f(0)h(t)+f(T)h(t-T)+f(2T)h(t-2T)\label{eqn:example for causality}
\end{equation}
이를 일반화하여 모든 실수 시간 $\tau$에 대해서 $y(t)$를 표현하면 다음과 같을 것이다.
\begin{equation}
    y(t) =\int_{-\infty}^{\infty}f(\tau)h(t-\tau)d\tau\label{eqn:convolution}
\end{equation}
이러한 특별한 적분 관계를 바로 콘볼루션(Convolution)\index{콘볼루션}이라고 한다. 어떤 책에서는 길쌈 혹은 길쌈곱이라고도 한다.
\par
이산 신호에 대해서는 다음과 같다.
\begin{equation}
    y[n]=\sum_{k=-\infty}^{\infty}f[k]g[n-k]\label{eqn:conv of disc}
\end{equation}
\paragraph{인과성}\label{인과성 해설} 여기서 잠깐 언급하고 넘어가고 싶은 게 있다. \pageref{causality} 페이지의 \ref{causality}에서 언급한 `인과성'을 기억하는가?
그 부분에서 인과성을 위한 조건이 식 \ref{eqn:causality condition}($h(t)=0\text{ for }t<0$)\이라고 하였다.
식 \ref{eqn:example for causality}\을 보고 왜 그런지 살펴보자.
지금이 $T$라고 하자. 지금까지 시스템에 들어온 입력들은 모두 $t<T$인 $t$일 때의 값들이다. 이 식에서는 $f(0)$만이 해당한다. $f(2T)$는 $T$ 이후에 들어올 입력이다.
따라서 현실적으로 이 시스템은 지금 $f(2T)$에 대해 반응할 수 없다. 이를 확실히 하는 조건이 바로 $f(2T)$에 곱해지는 $h(t-2T)\vert_{t=2T}=h(T-2T)=h(-T)$가 0이 되는 것이다.
일반화하면, $t<0$인 $t$에 대해서 $h(t)=0$이면, 즉 식 \ref{eqn:causality condition}이면 인과성이 보장된다.
\subsubsection{콘볼루션의 법칙과 성질} 
콘볼루션 연산에는 몇 가지 중요하고 유용한 법칙과 성질들이 있다. 하나씩 살펴보자.
\paragraph{결합법칙} 신호 $x_1$,$x_2$,$x_3$에 대해 다음 식이 항상 성립한다.
\begin{equation}
    x_1 * (x_2 * x_3)=(x_1*x_2)*x_3 \label{eqn:콘볼루션의 결합법칙}
\end{equation}
증명은 아래와 같다. 한번쯤 해 보고 대충 넘기자.
\begin{IEEEeqnarray*}{rCl}
    y(t)&=&(x_1*(x_2*x_3))(t)= x_1(t)* \left(\int_{-\infty}^{\infty} x_2(\tau_1)x_3(t-\tau_1) d\tau_1\right)\nonumber\\
    &=&\int_{-\infty}^{\infty} x_1(\tau_2) \int_{-\infty}^{\infty} x_2(\tau_1)x_3((t-\tau_2)-\tau_1)d\tau_1 d\tau_2\nonumber\\
    &=&\int_{-\infty}^{\infty}\int_{-\infty}^{\infty}x_1(\tau_2)x_2((\tau_1+\tau_2)-\tau_2)x_3((t-\tau_2)-\tau_1)d\tau_1d\tau_2\nonumber
\end{IEEEeqnarray*}
    $\tau_1+\tau_2=\tau_3$이라고 놓으면 $d\tau_1=d\tau_3$이므로
\begin{IEEEeqnarray*}{rCl}
    y(t)&=&\int_{-\infty}^{\infty}\int_{-\infty}^{\infty} x_1(\tau_2)x_2(\tau_3-\tau_2)x_3(t-\tau_3)d\tau_3d\tau_2\nonumber\\
    &=&\int_{-\infty}^{\infty}\int_{-\infty}^{\infty} x_1(\tau_2)x_2(\tau_3-\tau_2)d\tau_2x_3(t-\tau_3)d\tau_3\nonumber\\
    &=&\int_{-\infty}^{\infty}(x_1*x_2)(\tau_3)x_3(t-\tau_3)d\tau_3\nonumber
\end{IEEEeqnarray*}
\begin{equation*}
    \therefore (x_1*(x_2*x_3))(t)=((x_1*x_2)*x_3)(t)
\end{equation*}   

\paragraph{교환법칙} 신호 $x_1$과 $x_2$에 대해 다음 식이 항상 성립한다.
\begin{equation}
    x_1*x_2=x_2*x_1 \label{eqn:콘볼루션의 교환법칙}
\end{equation}
증명은 아래처럼 하면 된다. 역시 한번 대충 해보고 넘어가자.
\begin{IEEEeqnarray*}{rCl}
    y(t)&=&(x_1*x_2)(t)=\int_{-\infty}^{\infty}x_1(\tau_1)x_2(t-\tau_1)d\tau_1\nonumber
\end{IEEEeqnarray*}
    $t-\tau_1=\tau_2$라고 놓으면
    \begin{IEEEeqnarray*}{rCl}
    y(t)&=&\int_{-\infty}^{\infty}x_1(t-\tau_2)x_2(\tau_2)d\tau_2\nonumber
\end{IEEEeqnarray*}
\begin{equation*}
    \therefore (x_1*x_2)(t)=(x_2*x_1)(t)
\end{equation*}
\paragraph{분배법칙} 세 신호 $x_1$,$x_2$,$x_3$에 대해 다음 식이 항상 성립한다.
\begin{equation}
    x_1*(x_2+x_3)=x_1*x_2+x_1*x_3
\end{equation}
증명은 생략하겠다. 적분이 선형 연산임을 고려하면 쉽게 증명할 수 있다.
\paragraph{선형성} 콘볼루션은 다음과 같은 선형성을 만족한다.
\begin{equation}
    a(x_1*x_2)=(ax_1)*x_2=x_1*(ax_2)
\end{equation}
이 또한 적분이 선형 연산이므로 쉽게 증명할 수 있다.
\paragraph{공액 변환} 콘볼루션의 공액 변환(공액복소수)은 분배법칙과 비슷하게 작동한다.
\begin{equation}
    \overline{x_1*x_2} =\overline{x_1}*\overline{x_2}
\end{equation}
적분하는 변수는 실수 시간이므로 공액 변환의 영향을 받지 않는 것이다.
\paragraph{시간 이동}
콘볼루션한 결과 $y(t)$를 $T$만큼 시간이동한 것은 두 신호 중 어느 한 쪽을 $T$만큼 시간이동하여 콘볼루션한 것과 같다.
즉 이 시간이동 변환을 $S_T$라고 하면 아래 식이 성립한다.
\begin{equation}
    S_T(x_1*x_2)=(S_Tx_1)*x_2=x_1*(S_Tx_2)
\end{equation}
뒤에서 살펴보겠지만, 콘볼루션은 LTI 시스템에서 입력과 출력의 관계이다. 
이는 출력이 $T$만큼 지연되었다면 입력 또한 $T$만큼 지연되었다는 뜻이거나, 시스템이 $T$만큼 늦게 반응한다는 뜻이다. 증명은 생략하겠다.
\paragraph{미분} 콘볼루션 결과의 미분은 두 신호 중 어느 한 쪽을 미분하고 다른 쪽과 콘볼루션한 것과 같다. 위의 시간 이동과 같은 형태로 작용하는 것이다.
수식으로 쓰면 다음과 같다.
\begin{equation}
    \frac{d}{dt}(x_1*x_2)(t)=\left(\frac{dx_1}{dt}*x_2\right)(t)=\left(x_1*\frac{dx_2}{dt}\right)(t)
\end{equation}
증명은 다음과 같다.
\begin{IEEEeqnarray*}{lCl}
    \frac{d}{dt}(x_1*x_2)(t)&=&\int_{-\infty}^{\infty}x_1(\tau)\frac{d}{dt}x_2(t-\tau)d\tau\nonumber\\
    &=&\left(x_1*\frac{dx_2}{dt}\right)(t)\nonumber\\
    &=&\int_{-\infty}^{\infty}\frac{d}{dt}x_1(t-\tau)x_2(\tau)d\tau\nonumber\\
    &=&\left(\frac{dx_1}{dt}*x_2\right)(t)
\end{IEEEeqnarray*}
\paragraph{임펄스 함수와의 콘볼루션}
어떤 신호 $x(t)$와 임펄스 함수 $\delta(t)$를 콘볼루션하면 $x(t)$가 그대로 나온다. 즉
\begin{equation}
    x*\delta=x
\end{equation}
이다. 이 관계는 이미 보았을 것이다. 기억나는가? 바로 \pageref{eqn:f by impulse train} 페이지의 식 \ref{eqn:f by impulse train}이다.
이 성질은 뒤의 LTI 시스템 응답에서 중요하게 쓰일 것이다.
\paragraph{콘볼루션 결과의 길이}
$x_1$의 길이를 $l_1$, $x_2$의 길이를 $l_2$라고 하자. $x_1*x_2$의 길이는 $l_1+l_2$가 된다.
왜 그런가? $x_1$이 $x_2$와 콘볼루션하기 위해 적분을 시작할 때를 $t=0$이라 하자. 이 콘볼루션의 마지막 값은 $x_1$과 $x_2$ 모두 마지막 값일 때 나타날 것이다.
그리고, 길이가 $l_1$인 $x_1$의 각 값들은 각자 $x_2$의 길이 $l_2$만큼의 응답을 만들어낸다.
따라서 $x_2$의 마지막 값의 위치인 $t=l_2$에서 $x_1$의 마지막 값까지의 시간 $l_1$이 더해진 $t=l_1+l_2$일 때 마지막 값이 출력되는 것이다.
\subsection{콘볼루션과 LTI 시스템의 응답}
어떤 LTI 시스템이 입력 $x(t)$에 대해서 $H$라는 연산을 수행하여 출력 $y(t)$를 낸다고 하자. 즉 $y(t)=S(x(t))$이다.
이 관계는 \pageref{fig:system and io} 페이지의 \figurename~\ref{fig:system and io}에 나타나 있다.
그리고 이 시스템의 변환 $H$는 임펄스 함수 $\delta(t)$에 대하여 $h(t)$라는 응답을 내보낸다고 하자. 다시 말해 $h(t)=S(\delta(t))$이다.
이들로부터 이 시스템의 출력 $y(t)$를 어떻게 일반화하여 구할 수 있는지 알아보자.
먼저 식 \ref{eqn:f by impulse train}$\left(x(t)=\int_{-\infty}^{\infty}x(\tau)\delta(t-\tau)d\tau\right)$에서 시작하자.
이 $x(t)$ 표현을 시스템에 입력하여 $H$ 연산을 수행하자.
\begin{IEEEeqnarray*}{rCl}
    y(t)&=&H(x(t))=H\left(\int_{-\infty}^{\infty}x(\tau)\delta(t-\tau)d\tau\right)
\end{IEEEeqnarray*}
$H$는 선형성이 있으므로 가산성도 있고, 이에 따라 적분 안으로 들어갈 수 있다.
\begin{IEEEeqnarray*}{rCl}
    y(t)&=&\int_{-\infty}^{\infty}H\left(x(\tau)\delta(t-\tau)\right)d\tau
\end{IEEEeqnarray*}
$H$는 $t$의 함수에 작용하는 연산이고, $x(\tau)$는 $t$에 대해 상수이다. 그리고 $H$는 선형성에 의해 균질성도 갖고 있으므로 $x(\tau)$는 그냥 곱해진다.
\begin{IEEEeqnarray*}{rCl}
    y(t)&=&\int_{-\infty}^{\infty}x(\tau)H\left(\delta(t-\tau)\right)d\tau\nonumber\\
    &=&\int_{-\infty}^{\infty}x(\tau)h(t-\tau)d\tau\nonumber\\
    &=&\int_{-\infty}^{\infty}h(\tau)x(t-\tau)d\tau\IEEEyesnumber\label{eqn:response representation by filpped input}
\end{IEEEeqnarray*}
따라서 다음의 결론을 얻게 된다.
\begin{equation}
    y(t)=x(t)*h(t)\label{eqn:system response}
\end{equation}
\par
식 \ref{eqn:system response}\가 갖는 의미는 무엇일까? 어떤 LTI 시스템의 임펄스 응답만 안다면, 우리는 이 시스템에 어떤 입력이 들어가든 그 출력을 알 수 있다는 것이다.
이것이 가능한 이유는 당연하지만 LTI 성질을 만족하기 때문이다. 모든 신호는 각 순간의 신호값과 그 순간으로 평행이동한 임펄스 함수의 곱들이 무수히 늘어서 있는 것으로 간주할 수 있다(식 \ref{eqn:f by impulse train}).
선형성에 의해서 각 순간의 응답은 개별 입력들의 응답의 합이 되고, 개별 입력들의 응답은 각각 임펄스 응답에 각 순간의 신호값이 곱해진 것이다.
그리고 이 시스템은 시불변이므로 먼저 들어온 단위 임펄스이건 나중에 들어온 단위 임펄스이건 그만큼 시간 이동만 있을 뿐 같은 응답을 한다.
이것이 콘볼루션이 중요한 이유이다.
\subsection{콘볼루션에서 입력이 뒤집히는 이유}
이제 콘볼루션 수식 \ref{eqn:response representation by filpped input}에서 입력 신호가 시간에 대해 뒤집히는 이유를 현실 예제를 통해서 이해해 보자. 우주인이 지구에 `안녕하세요'라고 음성 인사를 보낸다. 그리고 지상의 직원이 그 음성을 듣는다고 하자.
이 `안녕하세요'라는 문장은 지상 직원에게 어떤 순서로 들릴까?
\begin{figure}
    \centering
    \image{why_signal_flipped.png}
    \caption{우주인이 보내는 음성 메시지는 지상 직원에게 어떤 순서로 들릴까?}\label{fig:why signal flipped}
\end{figure}
\figurename~\ref{fig:why signal flipped}\을 보라. `안'이 제일 먼저 들리니 시간이 지남에 따라서 오른쪽으로 이동한다. 그 다음 음절인 `녕'은 그 왼쪽에 위치하게 된다.
시간이 진행함에 따라 지상 직원의 귀로 들어가는 음절 순서는 결국 `요세하녕안' 이 된다. 이것이 입력이 시간에 대해 뒤집히는 이유이다.
이런 순서로 시스템에 입력된 신호는 어떤 순간 $t$에서 그 순간의 입력값에 대한 그 순간의 반응과 남아있는 앞서 입력된 신호들의 반응들이 모두 더해진 것이 그 순간의 총 반응이 되는 것이다.
\section{이산 시간 신호와 시스템에 대해}
지금까지 공부한 것들은 모두 연속 시간 신호와 시스템에 대한 얘기였다. 이산 시간 신호와 시스템에 대해서는 어떻게 될까?
$\int$는 $\sum$으로 바뀌고, 미분은 차분으로 바뀌고, 연속적인 시간 변수 $t$가 이산적인 시간 변수 $n$(단위 시간 간격 $T$는 보통 생략한다)\으로 바뀌는 차이가 있을 뿐이다.
추후 문서 버전업이 되면 이산 시간에 대해서도 다룰 수 있겠지만, 일단 지금은 연속 시간 신호와 시스템만 다루고 넘어가겠다.

\section{주파수 영역 해석}\label{chap:freq domain analysis}
이번 장에서는 시간 영역이 아니라 주파수 영역에서 신호와 시스템을 해석하는 법과 이에 따른 장점을 알아보자. 통신에서 주로 쓰일 연속 시간에 대해서만 다룰 것이다.
이산 시간에 대해서는 부록 \ref{apdx:disc time}\을 참고하라.
\subsection{오일러 공식}
본격적으로 신호를 다루기 전에 사인, 코사인과 복소지수함수의 관계에 대해서 알아보자. 이 함수들은 오일러 공식에 의해서 서로 연관된다.
오일러 공식이란 다음 항등식을 말한다.
\begin{equation}
    e^{j\omega t }=cos(\omega t)+jsin(\omega t)\label{eqn:Euler's formula}
\end{equation}
\pageref{사인파} 페이지의 \ref{사인파}에서 공부한 사인파를 기억하는가? 위 식의 실수부를 취하기만 하면 코사인으로 표현되는 사인파를 얻을 수 있다.
이것이 사인파를 주로 코사인을 이용해서 표현하는 이유이며, 복소수 $Ae^{j\left(\omega t+\phi\right)}$와 $Acos(\omega t+\phi)$은 동등하다.
이를 시각화한 것이 \figurename~\ref{fig:complex plane}\이다.
\begin{figure}
    \centering
    \image[height=6cm]{amplitude and phase.png}
    \caption{복소평면에서 나타낸 복소수 $Ae^{j\left(\omega t+\phi\right)}$}\label{fig:complex plane}
\end{figure}

\subsection{연속 시간 주기 신호의 주파수 해석}\label{CTFS}
이제 본격적으로 주파수 해석을 공부해보자. 먼저 연속 시간 주기 함수의 주파수 해석을 다루겠다. 이러한 신호의 주파수 해석은 연속 시간 푸리에 급수(Continuous Time Fourier Series;CTFS)\index{연속 시간 푸리에 급수}라고 한다.
\subsubsection{푸리에 급수 표현}
$T_0$\를 기본 주기로 갖는 신호 $f(t)$에 대해서 다음 식이 항상 성립한다.
\begin{equation}
    f(t)=f(t+T_0)
\end{equation}
\pageref{eqn:periodic signal} 페이지의 식 \ref{eqn:periodic signal}에서 보았듯이 이 함수는 $T_0$ 이외에도 모든 정수 $n$에 대해 $nT_0$들에 대해서도 주기적이다.
이러한 주기 신호의 중요한 특징은, 구간의 끝점은 잡기 나름이긴 한데, 어떤 실수 $a$에 대해서 구간 $[a,a+T_0)$ 의 함숫값만 알면 이 함수에 대해 모든 것을 알 수 있다는 것이다.
왜냐하면 이 밖의 구간은 그저 이 구간 내의 함숫값이 다시 나타날 뿐이기 때문이다.
한편, 이 신호는 기본 주파수 $f_0=\frac{1}{T_0}$ 의 성분 이외에 다른 주파수 성분들도 가질 수 있다.
$\frac{T_0}{2}$을 기본 주기로 갖는 함수 $g_2(t)=g_2\left(t+\frac{T_0}{2}\right)$는 모든 정수 $m$에 대해 $m\cdot \frac{T_0}{2}$ 을 주기로 가질 수 있으니, $g_2(t)$ 또한 $2\cdot \frac{T_0}{2}=T_0$에 대해서도 주기적이다.
$g_3(t)=g_3\left(t+\frac{T_0}{3}\right)$도 마찬가지로 $3\cdot \frac{T_0}{3}=T_0$에 대해서도 주기적이다.
\par
이를 일반화하면 $g_k(t)=g_k\left(t+\frac{T_0}{k}\right)$로 $\frac{T_0}{k}$를 기본 주기로 갖는 모든 $g_k(t)$들은 $T_0$에 대해 주기적이다.
이들 $g_k(t)$가 바로 $f(t)$를 이루는 성분들이 되고, 각 $g_k(t)$의 기본 주파수는 $\frac{T_0}{k}$의 역수인 $\frac{k}{T_0}=kf_0$이다. 이 주파수들이 $f(t)$를 이루는 성분 주파수들이다.
그렇다면 이 주파수 성분들은 어떤 함수이며 그 크기는 어떻게 구할까? 다시 \pageref{사인파} 페이지의 사인파에 대한 설명을 보자. 사인파는 단일 주파수만을 성분으로 갖고 있으므로 이들의 합을 이용해서 신호를 나타낼 수 있다. 
또한 식 \ref{eqn:Euler's formula}에서 사인파와 복소수는 동등하므로 최종적으로 $f(t)$는 아래와 같이 나타낼 수 있다.
\begin{equation}
    f(t)=\sum_{n=-\infty}^{\infty}c_n e^{j2\pi f_0 nt}=\sum_{n=-\infty}^{\infty}c_n e^{j\omega_0 nt}\label{eqn:fourier series representation}
\end{equation}
위 급수를 연속 시간 푸리에 급수\index{연속 시간 푸리에 급수}\index{CTFS}라고 부르며, $c_n$은 푸리에 계수라고 한다. 바로 이 $c_n$들이 $f(t)$의 주파수 성분들의 진폭이다.

\subsubsection{푸리에 급수 해석}
위의 $c_n$들을 어떻게 구할 수 있을까? 다음 과정을 천천히 따라가보자.
\\
먼저 식 \ref{eqn:fourier series representation}의 양변에 $e^{-j\omega_0kt}$를 곱한다.
\begin{IEEEeqnarray*}{rCl}
    f(t)e^{-j\omega_0kt}=\sum_{n=-\infty}^{\infty}c_n e^{j\omega_0nt}e^{-j\omega_0kt}
\end{IEEEeqnarray*}
양변을 주기 $T$ 구간에서 적분한다.
\begin{IEEEeqnarray*}{rCl}
    \int_{0}^{T}f(t)e^{-j\omega_0kt}dt&=&\int_{0}^{T}\sum_{n=-\infty}^{\infty}c_n e^{j\omega_0nt}e^{-j\omega_0kt}dt\nonumber\\
    &=&\sum_{n=-\infty}^{\infty}\int_{0}^{T}c_n e^{j\omega_0nt}e^{-j\omega_0kt}dt\nonumber\\
    &=&\sum_{n=-\infty}^{\infty}c_n\int_{0}^{T} e^{j\omega_0nt-j\omega_0kt}dt\nonumber\\
    &=&\sum_{n=-\infty}^{\infty}c_n\int_{0}^{T} e^{j\omega_0(n-k)t}dt\IEEEyesnumber\label{eqn:integration for fourier coef}
\end{IEEEeqnarray*}
우변의 $\int_{0}^{T} e^{j\omega_0(n-k)t}dt$를 구하기 위해 $k$값별로 경우를 나눠보자.
먼저 $n=k$인 경우이다.
\begin{IEEEeqnarray*}{rCl}
    \int_{0}^{T} e^{j\omega_0(n-k)t}dt&=&\int_{0}^{T} e^{(j\omega_0 0t)t}dt\nonumber=\int_{0}^{T} e^{0}d=\int_{0}^{T} 1 dt\nonumber\\
    &=&T
\end{IEEEeqnarray*}
다음으로 $n\neq k$인 경우에 대해 구하기 위해 오일러 공식을 전개하자.
\begin{IEEEeqnarray*}{rCl}
    \int_{0}^{T} e^{j\omega_0(n-k)t}dt&=&\int_{0}^{T} cos(\omega_0(n-k)t)dt+\int_{0}^{T} jsin(\omega_0(n-k)t)dt \nonumber\\
    &=&\int_{0}^{T} cos(\omega_0(n-k)t)dt+j\int_{0}^{T} sin(\omega_0(n-k)t)dt \IEEEyesnumber\label{eqn:Integration of Euler's formula when n!=k}
\end{IEEEeqnarray*}
그런데, $cos(\omega_0(n-k)t)$는 $T$인 구간 내에 코사인 한 주기 그래프가 $n-k$번 나타나는 함수이고, 코사인은 한 주기 구간에서 적분하면 $0$이 된다.
따라서 $T$ 구간에서 적분하면 $0\cdot (n-k)=0$이 된다. $sin(\omega_0(n-k)t)$도 마찬가지로 $T$ 구간에서 적분하면 0이 된다.
결과적으로 식 \ref{eqn:Integration of Euler's formula when n!=k}의 값은 $0$이다.
정리하면
\begin{IEEEeqnarray*}{rCl}
    \int_{0}^{T} e^{j\omega_0(n-k)t}dt=\begin{cases}
        T & \text{ if } n=k,\\
        0 & \text{ otherwise.}
    \end{cases}\label{eqn:Integration of Euler's formula on T}
\end{IEEEeqnarray*}
이를 식 \ref{eqn:integration for fourier coef}에 대입하여 계산하면 우변에서 $n\neq k$일 경우는 무조건 $0$이 나오므로 더해져도 의미가 없다. 따라서 $\sum$은 다음과 같이 정리된다.
\begin{equation*}
    \int_{0}^{T} f(t)e^{-j\omega_0nt}dt=Tc_n
\end{equation*}
$c_n$에 대해서 정리하면 최종적으로 다음 결과를 얻는다.
\begin{equation}
    c_n=\frac{1}{T}\int_{0}^{T} f(t)e^{-j\omega_0nt}dt\label{eqn:int interval of ctfs}
\end{equation}
\par
위에서 $[0,T)$ 구간에 대해 적분하는 것은 무슨 의미일까? 주기가 $T$이므로 길이가 $T$인 구간에 대해서 적분을 하면 그 신호의 특성을 완전히 뽑아낼 수 있다는 것이다.
그리고 한 주기 동안 적분하나 두 주기 동안 적분하나 정보량은 같겠지만 후자의 경우 적분 결과가 두 배가 될 것이니  나눠서 평균낸다고 생각하면 된다.

\subsubsection{푸리에 급수 해석 예제}
\figurename~\ref{fig:original periodic signal}\과 같은 주기가 $10$이고 기본 각주파수가 $\omega_0=\frac{2\pi}{10}=\frac{\pi}{5}$인 연속 시간 주기 신호 $f(t)$가 있다고 하자. 이 신호에 대해서 푸리에 계수를 구하면서 그 계수들로 그림을 그려 보자.
\begin{figure}
    \centering
    \image{original periodic signal.pdf}
    \caption{주기 신호 예}\label{fig:original periodic signal}
\end{figure}
\\
먼저 $c_0$을 구해보자. $e^{-j\omega_0 0t}=1$이므로 $c_0$을 구하는 것은 한 주기 내에서 $f(t)$를 적분하는 것과 같고, 이는 $f(t)$의 평균이기도 하다.
그래프에서 보면 $+1$과 $-1$이 번갈아 같은 길이만큼 나오므로 이 적분 결과는 $0$이므로 $c_0=0$이다.
\par
다음으로 $c_1$을 구해보자. $f(t)$는 시간에 따라 $+1$또는 $-1$이므로
\begin{IEEEeqnarray*}{rCl}
    c_1&=&\frac{1}{10}\int_{0}^{10} f(t)e^{-j\frac{\pi}{5}t}dt\\
    &=&\frac{1}{10}\int_{0}^{10} f(t)\left( cos\left(\frac{\pi}{5}t\right) -jsin\left(\frac{\pi}{5}t\right) \right) dt\\
    &=&\frac{1}{10}\left(\int_{0}^{5}1\left( cos\left(\frac{\pi}{5}t\right) -jsin\left(\frac{\pi}{5}t\right)\right)dt\right)\\
    &&+\frac{1}{10}\left(\int_{5}^{10} -1\left( cos\left(\frac{\pi}{5}t\right) -jsin\left(\frac{\pi}{5}t\right)\right)dt \right)
\end{IEEEeqnarray*}
위에서 코사인 적분은 각각 $0$이다. 홀수 $n$에 대해서는 전부 그럴 것이다(왜일까?).
한편 하나의 사인 적분은 반 주기 동안 적분하는 것이기 때문에 그 절댓값은 $2$가 된다. 위 식에 대입하면
\begin{IEEEeqnarray*}{rCl}
    c_1&=&\frac{1}{10}\left(\int_{0}^{5}-jsin\left(\frac{\pi}{5}t\right)dt + \int_{5}^{10} (-1)\cdot -jsin\left(\frac{\pi}{5}t\right)dt \right)\\
    &=&\frac{1}{10}\left(\int_{0}^{5}-jsin\left(\frac{\pi}{5}t\right)dt + \int_{5}^{10} jsin\left(\frac{\pi}{5}t\right)dt \right)\\
    &=&\frac{1}{10}\cdot \frac{5}{\pi}\left(-j\cdot 2 + j\cdot (-2) \right)=\frac{1}{10}\cdot \frac{5}{\pi} \cdot (-4j)=-\frac{2}{5}j\cdot \frac{5}{\pi}\\
    &=&-\frac{2}{\pi}j
\end{IEEEeqnarray*}
이다. 한편, $c_{-1}$은 $-c_1$이 될 것은 쉽게 알 수 있다.
이 두 계수 $c_1$과 $c_{-1}$로 신호를 합성하면 다음 식이 된다.
\begin{IEEEeqnarray*}{rCl}
    f_{1}(t)&=&\sum_{n=-1,1}c_n e^{j\omega_0nt}=c_1e^{j\omega_0t}+c_{-1}e^{-j\omega_0t}\\
    &=&\left(-\frac{2}{\pi}j\right)e^{j\omega_0t}+\left(\frac{2}{\pi}j\right)e^{-j\omega_0t}
\end{IEEEeqnarray*}
이 $f_1(t)$를 그래프로 그려보면 \figurename~\ref{fig:synthesization by c1 and c-1}이 된다.
\begin{figure}
    \centering
    \image[height=6cm]{c1.pdf}
    \caption{$c_1$과 $c_{-1}$로 그린 그림}\label{fig:synthesization by c1 and c-1}
\end{figure}
원래 신호와 그다지 닮아보이지 않는다. 하지만 좀 더 계산해보자. 다음은 $c_2$이다.
\begin{IEEEeqnarray*}{rCl}
    c_2&=&\frac{1}{10}\int_{0}^{10} f(t)e^{-j\frac{2\pi}{5}t}dt\nonumber\\
    &=&\frac{1}{10}\left(\int_{0}^{5}1\left( cos\left(\frac{2\pi}{5}t\right) -jsin\left(\frac{2\pi}{5}t\right)\right)dt\right)\\
    &&+\frac{1}{10}\left(\int_{5}^{10} -1\left( cos\left(\frac{2\pi}{5}t\right) -jsin\left(\frac{2\pi}{5}t\right)\right)dt \right)
\end{IEEEeqnarray*}
위 적분에서 코사인은 한 주기 단위로 적분하게 되니 $0$이 나올 것이다. 사인에 대해서도 마찬가지이므로 $c_2=0$이다.
이는 $\vert n \vert$이 짝수라면 계속 성립할 것이므로 앞으로 짝수 $n$에 대해서는 계산하지 않아도 된다.
\par
일반화하여 $n$이 양의 홀수일 때에 대한 계산을 해 보자.
\begin{IEEEeqnarray*}{rCl}
    c_n&=&\frac{1}{10}\int_{0}^{10} f(t)e^{-j\frac{n\pi}{5}t}dt\\
    &=&\frac{1}{10}\left(\int_{0}^{5}-jsin\left(\frac{n\pi}{5}t\right)dt + \int_{5}^{10} jsin\left(\frac{n\pi}{5}t\right)\right)dt\\
    &=&\frac{1}{10}\cdot \frac{5}{n\pi} \left ( -2j-2j\right)=\frac{1}{2}\cdot \frac{1}{n\pi}\cdot (-4j)\\
    &=&-\frac{2}{n\pi}j
\end{IEEEeqnarray*}
$n$이 음수라면 부호만 바꿔주면 된다.
이로부터 푸리에 계수 합성을 한 그래프들은 \figurename~\ref{fig:fourier_series_synthesization}\과 같다.
\begin{figure}[!hbp]
    \centering
    \image[height=18cm]{fourier_series_synthesization.pdf}
    \caption{푸리에 계수의 총 개수에 따른 합성 결과}\label{fig:fourier_series_synthesization}
\end{figure}
점점 원래 신호 $f(t)$에 가까워짐을 알 수 있다. 
신호의 분류에 따라 적용되는 푸리에 해석 방법은 달라지지만, 기본적으로 이것이 푸리에 해석을 하고 역으로 해석된 결과에서 원래 신호를 만들어내는 방법이다.

\subsubsection{CTFS의 성질}
신호로부터 푸리에 계수들을 얻는 변환을 $\mathcal{F}(\cdot )$으로 쓰자. 즉
\begin{equation*}
        \mathcal{F}(f(t))=\left\{c_n\right\}, n\text{은 정수}
\end{equation*}
이고, 이 의미를 다시 말하자면 시간 도메인 신호를 주파수 도메인의 수열로 바꾸는 것이다. 이 표기를 활용하여 $\mathcal{F}$ 변환의 성질을 알아보겠다.
\paragraph{선형성}
두 연속 시간 주기 신호 $f(t)$와 $g(t)$에 대해 다음 두 식이 성립한다.
\begin{IEEEeqnarray}{rCl}
    \mathcal{F}(af(t))=a\mathcal{F}(f(t)),a\text{ 는 복소수}\\
    \mathcal{F}(f(t)+g(t))=\mathcal{F}(f(t))+\mathcal{F}(g(t))
\end{IEEEeqnarray}
$\mathcal{F}(\cdot)$은 적분 연산이므로 선형성이 있고, 위 성질들은 당연히 성립한다. 어렵지 않으므로 증명은 생략하겠다.
\paragraph{시간 천이와 위상 천이}
어떤 신호 $f(t)$에 대해 시간 천이(이동)이 발생한다면 그 신호의 푸리에 계수들에는 위상 천이가 나타난다.
수식으로 증명하지 말고 그림으로 생각해보자. \pageref{fig:complex plane} 페이지의 \figurename~\ref{fig:complex plane}\을 보라.
이 그림 상에서 주어진 복소수 $e^{j\omega t+\phi}$의 위상각은 $\angle e^{j\omega t+\phi}=\omega t+\phi$이다.
다음으로 시간 $t_0$만큼 천이가 발생한다면 이 때의 복소수는 $e^{j\omega (t-t_0)+\phi}$이 되므로 위상각 또한 $\omega (t-t_0)+\phi=\omega t + \phi -\omega t_0$이 된다.
즉, 각주파수와 시간 천이량의 곱에 부호를 반전시킨 $-\omega t_0$이 위상 천이되는 양이다. 이 관계를 $\mathcal{F}(\cdot)$에 적용하여 수식으로 정리하면 다음과 같다.
\begin{equation}
    \mathcal{F}(f(t-t_0))=\left\{c_ne^{-j\omega_0 n t_0}  \right\}
\end{equation}
\paragraph{미분된 신호의 푸리에 급수 해석}
어떤 신호 $f(t)$의 미분인 $\frac{df(t)}{dt}$에 대해 푸리에 급수 해석을 하면 다음과 같다.
\begin{IEEEeqnarray*}{rCl}
    \mathcal{F}\left(\frac{df(t)}{dt}\right)&=&\mathcal{F}\left(\frac{d}{dt} \sum_{n=-\infty}^{\infty}c_n e^{j2\pi f_0 nt} \right)\\
    &=&\mathcal{F}\left(\sum_{n=-\infty}^{\infty}c_n \frac{d}{dt}e^{j2\pi f_0 nt} \right)\\
    &=&\mathcal{F}\left(\sum_{n=-\infty}^{\infty}c_n \cdot (j2\pi f_0n) \cdot e^{j2\pi f_0 nt} \right)\\
    &=&\mathcal{F}\left((j2\pi f_0n) \sum_{n=-\infty}^{\infty}c_n  e^{j2\pi f_0 nt} \right)\\
    &=&\left\{j2\pi f_0n\cdot c_n  \right \}\IEEEyesnumber\label{eqn:ctfs of diff}
\end{IEEEeqnarray*}
이 결과를 쉽게 기억하기 위해서는 다른 것들은 내버려두고 $e^{j2\pi f_0 nt}$부분만 시간에 대해 미분된다고 생각하면 된다.
또한 이 의미를 들여다보면 미분을 하게 되면 위상은 $90^\circ$ 빨라지고, 고주파수일수록 주파수에 비례하여 그 성분이 더 커진다는 것이다.
이는 후에 잡음과 관련하여 중요하게 될 결과이니 의미를 잘 기억해두자.
\paragraph{적분된 신호의 푸리에 급수 해석}
$\int_{-\infty}^{t}f(\tau)d\tau$의 푸리에 급수 해석 결과는 어떻게 될까?
증명이 그리 간단하지 않으므로 \pageref{eqn:ctfs of integration} 페이지의 식 \ref{eqn:ctfs of integration}만 봐도 되지만, 혹시나 증명을 원하시는 분들을 위해 다음과 같이 증명해보았다.
\\
어떤 정수 $n$에 대한 푸리에 계수 ${c'_n}$을 구해보자.
\begin{equation*}
    c'_n=\frac{1}{T}\int_{0}^{T} \left(\int_{-\infty}^{t}f(\tau)d\tau\right) e^{-j2\pi f_0 nt} dt
\end{equation*}
부분적분을 해주면($e^{-j2\pi f_0 nt}$를 적분하는 방법으로)
\begin{equation*}
    c'_n=\frac{1}{T}\left( \left.\int_{-\infty}^{t}f(\tau)d\tau \frac{1}{-j2\pi f_0 n} e^{-j2\pi f_0 nt} \right\vert_{t=0}^{t=T} -\int_{0}^{T} f(t)\frac{1}{-j2\pi f_0 n} e^{-j2\pi f_0 nt} dt \right)
\end{equation*}
왼쪽 항의 값을 계산해보면
\begin{multline*}
    \left.\int_{-\infty}^{t}f(\tau)d\tau \frac{1}{-j2\pi f_0 n} e^{-j2\pi f_0 nt} \right\vert_{t=0}^{t=T}\\
    =\frac{1}{-j2\pi f_0 n} \left( \int_{-\infty}^{T}f(\tau)d\tau e^{-j2\pi f_0 nT} - \int_{-\infty}^{0}f(\tau)d\tau \right)
\end{multline*}
여기서 $e^{-j2\pi f_0 nT}=cos(-j2\pi nf_0/f_0)-jsin(-j2\pi nf_0/f_0)=cos(-j2\pi n)-jsin(-j2\pi n)=1$이다. 대입하면
\begin{IEEEeqnarray*}{rCl}
    \left.\int_{-\infty}^{t}f(\tau)d\tau \frac{1}{-j2\pi f_0 n} e^{-j2\pi f_0 nt} \right\vert_{t=0}^{t=T}&=&\frac{1}{-j2\pi f_0 n} \left( \int_{-\infty}^{T}f(\tau)d\tau  - \int_{-\infty}^{0}f(\tau)d\tau \right)\\
    &=&\frac{1}{-j2\pi f_0 n} \int_{0}^{T}f(\tau)d\tau
\end{IEEEeqnarray*}
이다. $\int_{0}^{T}f(\tau)d\tau=0$인 경우 위 식의 값은 $0$이 되고, 남은 항에 대해 마저 계산하면
\begin{IEEEeqnarray*}{rCl}
    c'_n&=&-\frac{1}{T}\int_{0}^{T} f(t)\frac{1}{-j2\pi f_0 n} e^{-j2\pi f_0 nt} dt\\
    &=&\frac{1}{j2\pi f_0 n}\frac{1}{T}\int_{0}^{T} f(t)e^{-j2\pi f_0 nt}dt\\
    &=&\frac{1}{j2\pi f_0 n} c_n
\end{IEEEeqnarray*}
이 된다. $\mathcal{F}(\cdot)$을 이용해 표기하면 최종적으로
\begin{equation}
    \mathcal{F}\left(\int_{-\infty}^{t}f(\tau)d\tau\right)=\left\{\frac{1}{j2\pi f_0 n} c_n\right\}\label{eqn:ctfs of integration}
\end{equation}
이 된다. \pageref{eqn:ctfs of diff} 페이지의 미분의 푸리에 급수 결과 식 \ref{eqn:ctfs of diff}을 떠올려 보라.
미분과 적분은 서로 역연산 관계이니 미분에서 곱해지는 게 적분에선 나눠져야 할 것이다.
\par
$\int_{0}^{T}f(\tau)d\tau=0$이란 의미는 무엇일까? $f(t)$의 평균 즉 직류 성분이 $0$이란 소리이다.
$\int_{-\infty}^{t}f(\tau)d\tau$의 값이 수렴하기 위해서는 직류 성분이 $0$이 되어야 함은 당연하다.
안 그러면 이 직류 성분이 $-\infty$에서 $t$까지 적분되어 무한대의 값을 낼 것이기 때문이다.
\par
또한, 미분에서 고주파가 강조되었던 것과 반대로 적분에서는 저주파가 강조됨을 알 수 있다. 윗 문단의 내용도 이와 같이 생각하면 저주파 신호인 직류 성분이 무한히 강조된다고 생각하면 된다.
뒤에서 다시 설명할 기회가 있을 것 같지만, 이것이 적분기가 저대역 통과 필터와 비슷한 역할을 하는 이유이다. 고주파수로 갈수록 그 주파수 성분이 작아지게 만드는 것이 서로 비슷하다.
또한 열잡음은 언제 어디로 튈지 모르는 것이므로 빨리 변하고, 빨리 변한다는 것은 고주파 성분이 많다는 얘기이므로 적분기를 통과한 잡음은 작아지게 된다.
열잡음의 평균은 0인데, 평균에는 적분 혹은 합 연산이 들어감과 연관지어서 생각해봐도 좋다.
\paragraph{두 신호의 곱과 푸리에 급수 해석}
$T$에 대해 주기적인 두 신호 $f(t)$와 $g(t)$에 대해 $f(t)g(t)$도 $T$에 대해 주기적임은 쉽게 알 수 있다.
$f(t)$와 $g(t)$의 푸리에 계수 수열이 각각 $c_n$과 $d_n$일 때 $\mathcal{F}(f(t)g(t))$는 어떻게 되는지 알아보자.
\begin{IEEEeqnarray*}{rCl}
    \mathcal{F}(f(t)g(t))&=&\frac{1}{T} \int _{0}^{T} f(t)g(t) e^{-j2\pi f_0 nt} dt\\
    &=&\frac{1}{T}\int _{0}^{T} \sum_{k=-\infty}^{\infty} c_k e^{j2\pi f_0 kt}g(t)e^{-j2\pi f_0 nt} dt\\
    &=&\sum_{k=-\infty}^{\infty}c_k\cdot \frac{1}{T}\int _{0}^{T} g(t)e^{-j2\pi f_0 (n-k)t} dt\\
    &=&\sum_{k=-\infty}^{\infty}c_k d_{n-k}\\
    &=&\mathcal{F}(f(t))*\mathcal{F}(g(t))\IEEEyesnumber \label{eqn:fs of multi of periodic signals}
\end{IEEEeqnarray*}
마지막 줄은 \pageref{eqn:conv of disc} 페이지의 이산 신호 콘볼루션 식 \ref{eqn:conv of disc}\을 이용하였다.
\paragraph{두 신호의 콘볼루션과 푸리에 급수 해석}
$T$에 대해 주기적인 두 신호 $f(t)$와 $g(t)$에 대해 $\mathcal{F}(f(t)*g(t))$도 (아주 직관적이진 않지만) $T$에 대해 주기적임을 알 수 있다.
$f(t)$와 $g(t)$의 푸리에 계수 수열이 각각 $c_n$과 $d_n$일 때 $\mathcal{F}(f(t)*g(t))$이 어떻게 되는지 알아보자.
\begin{IEEEeqnarray*}{rCl}
    \mathcal{F}(f(t)*g(t))&=&\frac{1}{T}\int_{0}^{T}f(t)*g(t) e^{-j2\pi f_0 nt}dt\\
    &=&\frac{1}{T}\int_{0}^{T} \int_{0}^{T}f(\tau)g(t-\tau)d\tau \cdot e^{-j2\pi f_0 nt}dt\\
    &=&\frac{1}{T}\int_{0}^{T} \int_{0}^{T}f(\tau)g(t-\tau)d\tau \cdot e^{-j2\pi f_0 n(t-\tau + \tau)}dt\\
    &=&\frac{1}{T} \int_{0}^{T} \int_{0}^{T} f(\tau) e^{-j2\pi f_0 n(-\tau)}g(t-\tau) e^{-j2\pi f_0 n(t-\tau)} dt d\tau\\
    &=&\frac{1}{T} \int_{0}^{T} f(\tau) e^{-j2\pi f_0 n(-\tau)} \int_{0}^{T} g(t-\tau) e^{-j2\pi f_0 n(t-\tau)} dt d\tau
\end{IEEEeqnarray*}
$t-\tau=u$로 놓으면 $dt=du$이고 이를 대입하면
\begin{IEEEeqnarray*}{rCl}
    \mathcal{F}(f(t)*g(t))&=&\int_{0}^{T} f(\tau) e^{-j2\pi f_0 n(-\tau)}\frac{1}{T}\int_{0}^{T} g(u) e^{-j2\pi f_0 nu} du d\tau\\
    &=&\int_{0}^{T} f(\tau) e^{-j2\pi f_0 n(-\tau)}\mathcal{F}(g(t))d\tau\\
    &=&T\cdot \mathcal{F}(g(t))\cdot \frac{1}{T}\int_{0}^{T} f(\tau) e^{-j2\pi f_0 n(-\tau)}   d\tau\\
    &=&T\cdot \mathcal{F}(g(t))\cdot  \mathcal{F}(f(t))\\
    &=&T \mathcal{F}(f(t))\mathcal{F}(g(t)) \IEEEyesnumber \label{eqn:fs of conv of peridoc signals}
\end{IEEEeqnarray*}
위 식 \ref{eqn:fs of conv of peridoc signals}\과 식 \ref{eqn:fs of multi of periodic signals}를 같이 살펴보자.
시간 영역에서의 곱셈은 주파수 영역에서 콘볼루션이 되고, 시간 영역에서의 콘볼루션은 주파수 영역에서 곱셈이 된다.
\par
어떤 신호가 어떤 시스템에 입력되면 그 출력은 입력 신호와 시스템의 임펄스 응답의 콘볼루션이었다.
식 \ref{eqn:fs of conv of peridoc signals}이 의미하는 바는 주파수 영역에서 시스템의 반응을 알려면 입력의 푸리에 계수 수열과 시스템의 임펄스 응답의 푸리에 계수 수열을 곱해주면 된다는 것이다.
비록 푸리에 계수 수열을 구하고 역변환하는 과정이 필요하긴 하나, 비교적 머리아픈 콘볼루션 대신 곱셈으로 시스템의 반응을 알 수가 있는 것이다.

\subsubsection{Parseval의 정리}
Parseval의 정리는 다음과 같다.
\begin{equation}
    \frac{1}{T}\int_{0}^{T}(\vert f(t) \vert )^2dt=\sum_{n=-\infty}^{\infty} (\vert c_n\vert)^2\label{eqn:parseval}
\end{equation}
굳이 증명을 알 필요는 없고 이 결과와 \pageref{implication of parseval's thm} 페이지 마지막의 해석만 이해해도 되겠지만, 증명을 하고자 한다면 다음과 같이 할 수 있다.
\\먼저 푸리에 계수 합성식인 식\ref{eqn:fourier series representation}에서 시작하자.
\begin{equation*}
    f(t)=\sum_{n=-\infty}^{\infty}c_n e^{j2\pi f_0 nt}
\end{equation*}
한편, 어떤 복소수의 크기의 제곱은 그 복소수와 그 켤레복소수의 곱이다. 따라서 $f(t)$의 크기의 제곱은 $f(t)\overline{f(t)}$이다.
이 복소수 크기의 제곱과 푸리에 계수 합성식을 식 \ref{eqn:parseval}에 대입하면 다음과 같다.
\begin{IEEEeqnarray*}{rCl}
    \frac{1}{T}\int_{0}^{T}(\vert f(t) \vert )^2dt&=&\frac{1}{T}\int_{0}^{T}\left(f(t)\overline{f(t)} \right)^2dt\\
    &=&\frac{1}{T}\int_{0}^{T}\left(\sum_{n=-\infty}^{\infty}c_n e^{j2\pi f_0 nt}\right)\overline{\left(\sum_{m=-\infty}^{\infty}c_m e^{j2\pi f_0 mt} \right)} dt\\
    &=&\frac{1}{T}\int_{0}^{T}\left(\sum_{n=-\infty}^{\infty}c_n e^{j2\pi f_0 nt}\right)\left(\sum_{m=-\infty}^{\infty}\overline{c_m e^{j2\pi f_0 mt}} \right) dt\\
    &=&\frac{1}{T}\int_{0}^{T} \sum_{n=-\infty}^{\infty} \sum_{m=-\infty}^{\infty} \left(c_n \overline{c_m} e^{j2\pi f_0 nt}e^{-j2\pi f_0 mt}\right)dt\\
    &=&\frac{1}{T}\sum_{n=-\infty}^{\infty} \sum_{m=-\infty}^{\infty} c_n \overline{c_m}\int_{0}^{T} e^{j2\pi f_0(n-m)t}dt
\end{IEEEeqnarray*}
위 식의 결과에서 $m\neq n$이면 적분 결과는 $0$이 되고, $m=n$이면 적분 결과는 $T$가 된다. 이를 대입하면 다음과 같이 식 \ref{eqn:parseval}이 된다.
\begin{IEEEeqnarray*}{rCl}
    \frac{1}{T}\int_{0}^{T}(\vert f(t) \vert )^2dt&=&\frac{1}{T}\sum_{n=-\infty}^{\infty} c_n \overline{c_n}\cdot T\\
    &=&\sum_{n=-\infty}^{\infty} (|c_n|)^2
\end{IEEEeqnarray*}
직관적으로 이는 당연한 결과이다. 식 \ref{eqn:parseval}의 좌변은 $f(t)$의 전력을 의미한다. 그리고 각각의 푸리에 계수의 크기의 제곱은 그 주파수 성분의 전력을 의미한다.
즉 좌변은 전체 신호의 관점에서 전력을 구한 것이고 우변은 신호를 구성하는 성분들 각각의 전력을 구해서 더한 것이다.\label{implication of parseval's thm} 이 둘은 당연히 같을 것이다.



\subsection{연속 시간 비주기 신호의 주파수 해석}
연속 시간 주기 신호의 주파수 해석(CTFS)을 이용하여 보다 일반적인 연속 시간 비주기 신호의 주파수 해석을 해 보자.
이를 연속 시간 푸리에 변환(Continuous Time Fourier Transform;CTFT)\index{연속 시간 푸리에 변환}\index{CTFT}이라 한다.
\subsubsection{주기함수를 이용한 비주기 신호의 표현}
비주기 신호도 사실 주기 신호로 나타낼 수 있다. 어떤 비주기 신호 $f(t)$에 대해서 적당한 $T_0$마다 이 $f(t)$가 나타나는 주기 신호 $f_{T_0}(t)$를 생각하자.
이 때 $T_0\rightarrow \infty$라면 이 둘은 같아진다. 이렇게 함으로써 $\lim_{T_0\rightarrow \infty} f_{T_0}(t)$에 대해서 주파수 해석을 하면 이는 곧 $f(t)$에 대해 해석하는 것과 같아진다.
\begin{figure}
    \centering
    \image{aperiodic_and_periodic.pdf}
    \caption{원 비주기 신호 $f(t)$와 이에 대한 비주기 신호 $f_{T_0}(t)$들}
\end{figure}
\subsubsection{푸리에 변환 표현}
\pageref{eqn:fourier series representation} 페이지의 식 \ref{eqn:fourier series representation}에서 보았듯이 $f_{T_0}(t)$를 다음과 같이 쓸 수 있다.
\begin{equation}
    f_{T_0}(t)=\sum_{n=-\infty}^{\infty}c_ne^{j2\pi f_0nt}\label{eqn:ft0 representation of ctfs}
\end{equation}
이에 따른 $c_n$은 다음과 같이 구할 수 있었다.
\begin{equation*}
    c_n=\frac{1}{T_0}\int_{0}^{T_0}f_{T_0}(t)e^{-j2\pi f_0nt}dt
\end{equation*}
다음과 같이 $F_{T_0}(f_0)$를 정의하자.
\begin{equation}
    F_{T_0}(f_0)\equiv T_0c_n=\int_{0}^{T_0}f_{T_0}(t)e^{-j2\pi f_0nt}dt\label{eqn:FT0 on o to T0}
\end{equation}
이를 식 \ref{eqn:ft0 representation of ctfs}에 대입하자.
\begin{equation*}
    f_{T_0}(t)=\sum_{n=-\infty}^{\infty} F_{T_0}(f_0) e^{j2\pi f_0nt}\frac{1}{T_0}
\end{equation*}
양변에 $\lim_{T_0 \rightarrow \infty}$을 취하면 좌변은 원래 함수가 되고 우변의 급수는 정적분으로 바뀐다.
또한 기본 주파수의 배수였던 $f_0n$들은 그냥 연속적인 $f$로 바뀐다.
\\
$F(f)\equiv \lim_{T_0 \rightarrow \infty}F_{T_0}(f_0)$라고 정의하면 
\begin{equation}
    \lim_{T_0 \rightarrow \infty}f_{T_0}(t)=f(t)=\int_{-\infty}^{\infty}F(f)e^{j2\pi ft}df\label{eqn:ctft}
\end{equation}
이다. 한편, 식 \ref{eqn:FT0 on o to T0}에서 적분구간은 반드시 $[0,T_0)$로 할 필요는 없다. 어떤 구간이건 길이가 $T_0$이기만 하면 함수의 정보를 충분히 얻어낼 수 있기 때문이다.
이를 고려하여 적분구간을 $[-T_0/2,T_0/2)$로 바꾸자. 그렇게 하면 $F(f)$는 다음과 같이 된다.
\begin{IEEEeqnarray}{rCl}
    F(f)&=&\lim_{T_0 \rightarrow \infty} F_{T_0}(f_0)=\lim_{T_0 \rightarrow \infty} \int_{-\frac{T_0}{2}}^{\frac{T_0}{2}}f_{T_0}(t)e^{-j2\pi f_0nt}dt\nonumber\\
    &=&\int_{-\infty}^{\infty}f(t)e^{-j2\pi ft}dt\label{eqn:inverse ctft}
\end{IEEEeqnarray}
이렇게 연속 시간 비주기 신호 $f(t)$와 이의 푸리에 변환인 $F(f)$, 그리고 $F(f)$에서 역변환을 통하여 $f(t)$를 구하는 방법을 알게 되었다.
\par
CTFS의 경우(식 \ref{eqn:int interval of ctfs})와 다르게 여기선 적분을 실수 전체에 대해서 하고 있다. 비주기 신호이기 때문에 모든 시간 영역 전체에 정보가 분포하기 때문에 그런 것이다.
그리고 당연하게도 적분구간의 길이인 무한대로 나누진 않고 있는데, 이 나눗셈은 적분 앞에 붙는 게 아니라 $T_0$이 $\infty$로 가는 극한, 즉 $\sum$이 $\int$로 바뀌면서 안쪽에 들어간 것이라고 생각하면 된다.

\subsubsection{CTFT의 성질}
\paragraph{선형성}
CTFT에도 마찬가지로 선형성이 있다.
\begin{equation}
    \mathcal{F}(af(t)+bg(t))=a\mathcal{F}(f(t))+b\mathcal{F}(g(t))
\end{equation}
\paragraph{쌍대성(Duality)}\index{쌍대성}\index{Duality}
쌍대성은 대칭성(Symmetry)라고도 한다. 
식 \ref{eqn:ctft}\와 식 \ref{eqn:inverse ctft}\를 보라. 시간 변수 $t$와 주파수 변수 $f$를 맞바꾸되, 둘 중 하나에만 $-$를 붙여주면 된다.
간단히 정리하면 아래와 같다.
\begin{IEEEeqnarray}{rCl}
    f(t)&\Longleftrightarrow& F(f)\\
    F(t)&\Longleftrightarrow& f(-f)
\end{IEEEeqnarray}

\paragraph{시간 천이와 위상 천이}
CTFT에서도 마찬가지로 성립한다.
\begin{equation}
    \mathcal{F}(f(t-t_0))=F(f)e^{-j2\pi ft_0}
\end{equation}
\paragraph{시간 비율 변화에 따른 푸리에 변환의 변화}
시간 변수 $t$에 $a$라는 값이 곱해져서 $at$가 되는 것은 것은 무슨 의미일까?
예륻 들어 $a=3$이라 하자. 시간 변수가 $t$였을 때 입력이 $1$이 되기 위해서는 그냥 $t=1$이면 되었다. 그런데 $3t$에 대해서는 어떤가? $t=1/3$이면 같은 입력이 된다.
이는 성분 주파수가 $3$배로 빨라졌다는 것을 뜻한다. 기존에는 주파수 $f$ Hz/s의 성분 값이 $F(f)$였는데, 이제는 이 값이 주파수 $3f$ Hz/s에 대응되어야 하는 것이다.
수식으로 나타내면 푸리에 해석 결과가 $F\left(\frac{f}{3}\right)$가 됨을 뜻한다. 주파수에 $3$이 붙었으니 입력될 때는 $3$으로 나눠주는 것이다.
여기서 끝이 아니다. $3$이 곱해짐에 따라 시간 영역에서 신호 $f$가 차지하는 영역은 위에서 설명했듯이 $1/3$배로 줄었고, 이는 $\vert f \vert ^2$의 넓이(총 에너지) 또한 $1/3$배가 되었다는 뜻이다.
따라서, 식 \ref{eqn:Parseval's thm on ctft}에 의해서 푸리에 변환 결과의 크기의 제곱의 넓이 또한 $1/3$배로 줄어야 한다.
$F\left(\frac{f}{3}\right)$이란 것은 주파수 영역에서 푸리에 변환이 차지하는 범위가 시간에서와는 반대로 $3$배 늘어났다는 의미이다. 결국 푸리에 변환 결과의 높이는 $1/3$이 되어야 $(1/3)^2\cdot 3=1/3$이 된다.
이를 수식으로 정리하면 다음과 같다.
\begin{equation}
    \mathcal{F}(f(at))=\frac{1}{\vert a \vert} F\left(\frac{f}{a}\right)
\end{equation}
\paragraph{미분된 신호의 푸리에 변환}
CTFS(식 \ref{eqn:ctfs of diff})에서와 비슷하게 어떤 어떤 신호 $f(t)$의 미분인 $\frac{df(t)}{dt}$에 대해 푸리에 급수 해석을 하면 다음과 같다.
\begin{equation}
    \mathcal{F}\left( \frac{df(t)}{dt} \right) = j2\pi ft F(f)
\end{equation}

\paragraph{두 신호의 곱과 푸리에 변환}
두 신호가 곱해진 신호의 CTFT는 다음과 같다.
\begin{equation}
    \mathcal{F}(f(t)g(t))=\mathcal{F}(f(t))*\mathcal{F}(g(t))
\end{equation}

\paragraph{콘볼루션과 푸리에 변환}
두 신호의 콘볼루션을 CTFT하면 다음과 같다.
\begin{equation}
    \mathcal{F}(f(t)*g(t))=\mathcal{F}(f(t))\mathcal{F}(g(t))
\end{equation}
\subsubsection{Parseval의 정리}
비주기 신호는 전력 대신 에너지를 사용한다.(왜일까?)
이에 따라 비주기 신호에 대한 Parseval의 정리는 Rayleigh 에너지 정리라고도 불린다.
수식으로 표현하면 다음과 같다.
\begin{equation}
    \int_{-\infty}^{\infty}\left(\vert f(t) \vert \right)^2 dt= \int_{-\infty}^{\infty}\left(\vert F(f) \vert \right)^2 df\label{eqn:Parseval's thm on ctft}
\end{equation}



\subsubsection{중요한 CTFT 결과}

\paragraph{$1$의 CTFT}
델타 함수의 정의였던 식 \ref{eqn:def of delta}\을 기억하는가? 이 정의에서 $1$의 푸리에 변환은 바로 나온다.
\begin{IEEEeqnarray*}{rCl}
    \mathcal{F}(1)&=&\int _{-\infty}^{\infty}1 \cdot e^{-j2\pi ft}dt\\
    &=&\delta(f)\IEEEyesnumber\label{eqn:ctft of 1}
\end{IEEEeqnarray*}
식 \ref{eqn:ctft of 1}의 의미는 무엇인가? 주파수 해석을 하고 있다는 것을 상기하자. $y(t)=1$이란 신호의 주파수는 얼마인가? 바로 $0$이다.
즉 위 결과는 $y(t)=1$이란 신호의 주파수 성분은 $f=0$에만 존재한다, 다르게 말하면 집중되어 있다고 해석할 수 있다.

\paragraph{$e^{j2\pi f_0t}$의 CTFT}
위의 결과를 일반화하면 어떻게 될 것인가? $e^{j2\pi f_0t}$라는 신호의 성분 주파수는 $f=f_0$ 뿐이다.
따라서 이 신호를 푸리에 변환하면 그 결과는 $\delta(f-f_0)$이 됨은 쉽게 알 수 있다. 수식적인 증명은 직접 해보시라.
또한 이를 $1$이란 신호의 주파수 대역을 $f_0$로 옮긴 것으로 해석하면 $\mathcal{F}(f(t))=F(f)$에 대해 $\mathcal{F}(f(t)e^{j2\pi f_0t})=F(f-f_0)$이란 일반화도 가능할 것이다.
\paragraph{사인파의 CTFT}
사인이나 코사인 신호를 주파수 변환하면 어떻게 될까?각각을 $e^{j2\pi f_0t}$ 함수를 이용해서 표현해보면 쉽게 알 수 있을 것이다.
\paragraph{$\delta(t)$의 CTFT}
식 \ref{eqn:ctft of 1}을 반대로 생각해보면 $\delta(t)$를 푸리에 변환하면 $1$이 됨을 알 수 있다.
푸리에 변환 결과가 $1$이란 것은 또 무슨 소리인가? 바로 모든 주파수 성분이 같은 크기로 균일하게 존재한다는 의미이다.
무한히 많은 주파수 성분들이 다 더해지니 델타 함수처럼 극단적으로 빨리 변하는 함수가 나타난 것이다.
\par
또한, 시스템의 임펄스 응답을 알면 어떤 입력이든지 그에 따른 출력을 알 수 있다는 것의 주파수 관점 설명도 이로부터 할 수 있다.
델타 함수는 모든 주파수 성분을 균일하게 $1$의 크기로 담고 있으니 시스템에 델타 함수를 입력한다는 것은 크기가 $1$인 모든 주파수 성분을 입력해본다는 뜻이다.
따라서 델타 함수의 응답을 알면 시스템을 완전히 알 수 있는 것이다.
\paragraph{$u(t)$의 CTFT}
단위계단함수의 푸리에 변환은 생각보다 쉽지 않을 것이다. 천천히 따라가볼 사람은 따라가고 아닌 사람은 그냥 결과만 알고 있자. 적분된 신호의 푸리에 변환을 구하기 위해 필요하니 알긴 해야한다.
\\
먼저 $u(t)=\frac{1}{2}sgn(t)+\frac{1}{2}$이다. 여기서 $sgn(t)$는 `시그넘'이라고 읽으며 입력 $t$에 대한 부호를 알려주는 함수이다.
부호는 `사인'이지만 이러면 sine과 겹치기 때문에 signum이라고 이름붙었다. 이를 이용하자.
\begin{IEEEeqnarray*}{rCl}
    \mathcal{F}(u(t))&=&U(f)=\mathcal{F}\left(\frac{1}{2}sgn(t)+\frac{1}{2}\right)\\
    &=&\frac{1}{2}\mathcal{F}(sgn(t))+\frac{1}{2}\delta(f)\IEEEyesnumber\label{eqn:ctft of sgn}
\end{IEEEeqnarray*}
위 식에서 $\mathcal{F}(sgn(t))$를 구해야 한다. 이를 구하기 위해선 약간의 기교가 필요하다.
\begin{IEEEeqnarray*}{rCl}
    \mathcal{F}(sgn(t))&=&\int_{-\infty}^{\infty}sgn(t)e^{-j2\pi ft}dt\\
    &=&\int_{-\infty}^{0}-e^{-j2\pi ft}dt + \int_{0}^{\infty}e^{-j2\pi ft}dt\\
    &=&\int_{0}^{-\infty}e^{-j2\pi ft}dt+ \int_{0}^{\infty}e^{-j2\pi ft}dt\\
    &=&\left.\frac{1}{-j2\pi f}e^{-j2\pi ft}\right\vert_{t=0}^{t=-\infty}+\left.\frac{1}{-j2\pi f}e^{-j2\pi ft}\right\vert_{t=0}^{t=\infty}\\
\end{IEEEeqnarray*}
여기서, $-\infty$를 대입한 값과 $\infty$를 대입한 값은 부호만 반대이니 상쇄될 것이고(엄밀하진 않으나 코시 주값이라고 생각하자) 남는 것은 $0$을 대입한 값 2개이다. 따라서
\begin{equation*}
    \mathcal{F}(sgn(t))=-2\cdot \frac{1}{-j2\pi f}=\frac{1}{j\pi}
\end{equation*}
이를 식 \ref{eqn:ctft of sgn}에 대입하자.
\begin{IEEEeqnarray*}{rCl}
    U(f)&=&\frac{1}{2}\mathcal{F}(sgn(t))+\frac{1}{2}\delta(f)\\
    &=&\frac{1}{j2\pi f}+\frac{1}{2}\delta(f)\IEEEyesnumber\label{eqn:ctft of u}
\end{IEEEeqnarray*}
이로써 단위 계단 함수의 푸리에 변환을 구했다.
\paragraph{구형 함수의 CTFT}
구형 함수 $rect(t)$ 혹은 $\sqcap (t)$의 CTFT는 이 함수를 단위 계단 함수로 표현함으로써 구할 수 있다.
\begin{equation}
    \sqcap(t)=u\left(t+\frac{1}{2}\right)-u\left(t-\frac{1}{2}\right)
\end{equation}
식 \ref{eqn:ctft of u}\을 이용해보자.
\begin{IEEEeqnarray*}{rCl}
    \mathcal{F}(\sqcap(t))&=&\mathcal{F}\left(u\left(t+\frac{1}{2}\right)\right)-\mathcal{F}\left(u\left(t-\frac{1}{2}\right)\right)\\
    &=&\left(\frac{1}{j2\pi f}+\frac{1}{2}\delta(f)\right)\left(e^{j2\pi f (1/2)t}-e^{j2\pi f (-1/2)t}\right)\\
\end{IEEEeqnarray*}
이 때, $f\neq 0$이면 $\delta(f)$=0이고, $f=0$이면 우변이 $1-1=0$이므로 이러나저러나 델타 함수 항은 의미가 없어진다.
또한 오른쪽 괄호의 값은 오일러 공식에 넣어보면 $j2sin(\pi ft)$이다. 따라서
\begin{IEEEeqnarray*}{rCl}
    \mathcal{F}(\sqcap(t))&=&\frac{1}{j2\pi f}\cdot j2sin(\pi ft)\\
    &=&\frac{1}{\pi f}sin(\pi ft)\\
    &=&sinc(t)\IEEEyesnumber\label{eqn:ctft of rect is sinc}
\end{IEEEeqnarray*}
이란 결론이 나온다. 특히 이는 sinc 함수의 정의이기도 하니 잘 알아두자. 이 식은 추후 디지털 통신에서 중요하게 쓰일 것이다.

\paragraph{적분된 신호의 푸리에 변환} 
$f(t)=\int_{-\infty}^{t}g(\tau)d\tau$이고 $\mathcal{F}(g(t))=G(f)$일 때 $\mathcal{F}(f(t))$가 어떻게 되는지 알아보자.
먼저 $f(t)$를 단위 계단 함수를 이용해서 표현하자. $g(\tau)$가 $\tau=t$까지만 존재하게 하고 실수 전 구간에 대해 적분하면 되므로
\begin{IEEEeqnarray*}{rCl}
    f(t)&=&\int_{-\infty}^{t}g(\tau)d\tau\\
    &=&\int_{-\infty}^{\infty}g(\tau)u(-(\tau-t)) d\tau\\
    &=&\int_{-\infty}^{\infty}g(\tau)u(t-\tau)d\tau\IEEEyesnumber
\end{IEEEeqnarray*}
이다.
이는 $g(t)*u(t)$이므로 푸리에 변환을 하면 그 결과는 $\mathcal{F}(f(t))=F(f)=\mathcal{F}(g(t)*u(t))=G(f)U(f)$이다.
식 \ref{eqn:ctft of u}에서 구한 결과를 이용하면
\begin{equation*}
    F(f)=\frac{G(f)}{j2\pi f}+\frac{G(f)}{2}\delta(f)
\end{equation*}
이 된다. 두 번째 항은 델타 함수에 의해서 $f=0$일 때에만 의미있는 항이 되므로(식 \ref{eqn:f value calc with delta}) $G(f)$를 $G(0)$으로 바꾸면 최종적으로
\begin{equation}
    \mathcal{F}\left(\int_{-\infty}^{t}g(\tau)d\tau\right)=\frac{G(f)}{j2\pi f}+\frac{G(0)}{2}\delta(f)
\end{equation}
이 된다.
\par
지금까지 기초적인 신호와 시스템에 대해서 알아보았다. 다음 장에서는 확률에 대해서 알아보자.
\include{probability_and_random_process}
\chapter{기초 아날로그 통신}

\section{진폭 변조}

\section{각 변조}
\chapter{기초 디지털 통신}

\section{아날로그-디지털 변환}

\section{디지털 변조}

\section{기저대역 통신}
\include{wireless}
\chapter{기초 정보이론}

\appendix
\include{disc_time}
\printindex

\end{document}