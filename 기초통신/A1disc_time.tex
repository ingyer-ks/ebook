\chapter{이산 시간 신호의 주파수 해석}\label{apdx:disc time}
\ref{chap:freq domain analysis}장에서 다루지 않은 이산 시간 신호의 주파수 해석에 대해서 간단히 다뤄보겠다.
\section{이산 시간 주기 신호의 주파수 해석}
먼저 이산 시간 주기 신호에 대한 주파수 해석인 이산 시간 푸리에 급수(Discrete Time Fourier Series;DTFS)에 대해 알아보자. 많은 부분이 CTFS와 비슷하므로 대충 넘어가겠다.
\subsection{푸리에 급수 표현}
$N$을 기본 주기로 갖는 이산 신호 $f[n]$에 대해 다음 식이 항상 성립한다.
\begin{equation}
    f[n]=f[n+N]
\end{equation}
\pageref{CTFS} 페이지에서 보았던 연속 시간 주기 함수의 특징이 이산 시간 주기 신호에도 유사하게 적용된다.
단, 연속 시간의 경우 성분 주파수들은 기본 주파수의 배수이기만 하면 무한히 커질 수 있었다. 그러나 이산 시간의 경우에는 그렇지 않다.
왜냐하면 연속 시간에서는 시각들 사이에 간격이란 게 없었으나, 이산 시간에서는 한 시점과 다음 시점 사이에 간격이 존재하기 때문이다.
수식을 통해서 생각해보자.
식 \ref{eqn:fourier series representation}\과 유사하게 이산 시간 신호를 표현해보자.
\begin{equation*}
    f[n]=\sum _{k} c_k e^{j2\pi F_0kn}
\end{equation*}
여기서 $F_0=\frac{1}{N}$이다. 연속 시간에서의 주파수와 구분하기 위하여 대문자를 사용했다. 이를 지수함수에 대입하고 $k$를 1부터 하나씩 증가시켜보자.
\begin{IEEEeqnarray*}{c}
    e^{j2\pi 0\cdot n/N}=e^{0}=1,e^{j2\pi n/N},e^{j2\pi 2n/N},\cdots ,e^{j2\pi (N-1)n/N},\\
    e^{j2\pi nN/N}=e^{j2\pi n}=1,e^{j2\pi n(N+1)/N}=e^{j2\pi n}e^{j2\pi n/N}=e^{j2\pi n/N},\cdots 
\end{IEEEeqnarray*}
위에서 볼 수 있듯이, $k$는 $0$부터 $N-1$까지만 의미가 있다. 이를 넘어서게 되면 다시 같은 주파수 성분이 반복될 뿐이다.
따라서, 이산 시간 주기 신호의 푸리에 급수 표현은 다음과 같다.
\begin{equation}
    f[n]=\sum_{k=0}^{N-1}c_ke^{j2\pi F_0kn}=\sum_{k=0}^{N-1}c_ke^{j\Omega_0kn}\label{eqn:DTFS}
\end{equation}
이를 이산 시간 푸리에 급수\index{이산 시간 푸리에 급수}\index{DTFS}라고 한다. $F_0$이라 쓴 것과 마찬가지로 연속 시간 각주파수와 구분하기 위해 $\Omega_0=\frac{2\pi}{N}$을 사용하였다.
\subsection{푸리에 급수 해석}
위의 $c_k$는 어떤 방법으로 계산할 수 있을까? 연속 시간 주기 신호에 대해 했던 것과 비슷하게 시작하자. 
\\
먼저 식 \ref{eqn:DTFS}의 양변에 $e^{-j\Omega_0ln}$을 곱한다.
\begin{IEEEeqnarray*}{rCl}
    f[n]e^{-j\Omega_0ln}&=&\sum_{k=0}^{N-1}c_ke^{j\Omega_0kn}e^{-j\Omega_0ln}\\
    &=&\sum_{k=0}^{N-1}c_ke^{j\Omega_0(k-l)n}
\end{IEEEeqnarray*}
양변에서 주기 $N$ 구간에 대해 급수합을 구한다.
\begin{IEEEeqnarray*}{rCl}
    \sum_{n=0}^{N-1}f[n]e^{-j\Omega_0ln}&=& \sum_{n=0}^{N-1}\sum_{k=0}^{N-1}c_ke^{j\Omega_0(k-l)n}\\
    &=&\sum_{k=0}^{N-1}c_k\sum_{n=0}^{N-1}e^{j\Omega_0(k-l)n}\IEEEyesnumber\label{eqn:inner product of f and basis in disc}
\end{IEEEeqnarray*}
다음으로 $\sum_{n=0}^{N-1}e^{j\Omega_0(k-l)n}$을 계산해보자.
먼저 $k=l$일 경우엔 $N$이 됨을 바로 알 수 있다. 다음으로 $k\neq l$일 경우에 대해 계산하자.
편의를 위해 $r=e^{j\Omega_0(k-l)}$이라고 놓자. 등비수열의 급수 공식을 이용하면 다음과 같다.
\begin{IEEEeqnarray*}{rCl}
    \sum_{n=0}^{N-1}e^{j\Omega_0(k-l)n}&=&\sum_{n=0}^{N-1}r^n\\
    &=&\frac{1-r^N}{1-r}=\frac{1-e^{j\Omega_0(k-l)N}}{1-r}=\frac{1-e^{j2\pi \frac{(k-l)N}{N}}}{1-r}\\
    &=&\frac{1-e^{j2\pi {(k-l)}}}{1-r}=\frac{1-1}{1-r}\\
    &=&0
\end{IEEEeqnarray*}
식 \ref{eqn:inner product of f and basis in disc}에 대입하면 다음과 같다.
\begin{equation*}
    \sum_{n=0}^{N-1}f[n]e^{-j\Omega_0ln}=c_lN
\end{equation*}
양변을 $N$으로 나눠서 최종적으로 $c_l$는 다음 식으로 구할 수 있다.
\begin{equation}
    c_l=\frac{1}{N}\sum_{n=0}^{N-1}f[n]e^{-j\Omega_0ln}
\end{equation}

\subsection{DTFS의 성질}
연속 시간이냐 이산 시간이냐에 따라 약간의 차이는 있으나 크게는 CTFS의 성질들과 같은 성질을 갖는다.
CTFS에서와 마찬가지로 $\mathcal{F}(\cdot)$을 이산 시간 주기 신호의 DTFS 계수를 얻는 변환이라고 하자.
DTFS 변환 $\mathcal{F}(\cdot)$에도 다음과 같은 성질들이 있다. 되도록 증명하지 않고 그냥 넘어가겠다.
\paragraph{선형성}
다음과 같은 성질이 있다.
\begin{IEEEeqnarray}{c}
    \mathcal{F}(af[n])=a\mathcal{F}(f[n])\\
    \mathcal{F}(f[n]+g[n])=\mathcal{F}(f[n])+\mathcal{F}(g[n])
\end{IEEEeqnarray}
\paragraph{시간 천이와 위상 천이}
시간 천이와 위상 천이에는 다음과 같은 관계가 있다.
\begin{equation}
    \mathcal{F}(f[n-n_0])=\left\{ c_k e^{-j2\pi F_0kn_0} \right\}
\end{equation}
\paragraph{누적 신호의 푸리에 해석}
$\mathcal{F}(f[n])=c_k$에 대해서 $f[n]$을 누적시킨 신호에 대해서 다음이 성립한다.
\begin{equation}
    \mathcal{F}\left(\sum_{l=0}^{n}f[l]\right)=\left\{\frac{1}{j2\pi F_0k}c_k\right\}
\end{equation}
\paragraph{두 신호의 곱과 푸리에 급수 해석}
두 신호의 곱셈에 대해 푸리에 급수 해석을 하면 다음과 같다. 주파수 범위가 유한하므로 이런 것이다.
\begin{IEEEeqnarray*}{rCl}
    \mathcal{F}(f[n]g[n])&=&\mathcal{F}\left(\sum_{k=0}^{N-1}c_ke^{j\Omega_0kn}\sum_{m=0}^{N-1}d_me^{j\Omega_0mn}\right)\\
    &=&\mathcal{F}\left(\sum_{k=0}^{N-1}\sum_{m=0}^{N-1}c_kd_me^{j\Omega_0(k+m)n} \right)\\
    &=&\sum_{k=0}^{N-1}\sum_{m=0}^{N-1}c_kd_m\mathcal{F}\left(e^{j\Omega_0(k+m)n}\right)\\
    &=&\sum_{k=0}^{N-1}\sum_{r=k}^{k+N-1}c_kd_{r-k}\mathcal{F}\left(e^{j\Omega_0rn}\right)\\
    &=&\sum_{k=0}^{N-1}\sum_{r=k}^{k+N-1}c_kd_{r-k}\delta(l-r)\\
    &=&\sum_{k=0}^{N-1}c_kd_{l-k}\\
    &=&\mathcal{F}(f[n]) \circledast \mathcal{F}(g[n])\IEEEyesnumber
\end{IEEEeqnarray*}
$\circledast$기호는 원형 콘볼루션(Cyclic Convolution, Circular Convolution)\index{원형 콘볼루션}\index{Cyclic Convolution}\index{Circular Convolution}을 나타낸다.
이산 시간에서는 가능한 성분 주파수들이 한정적이니(이 한정된 성분 주파수들이 반복될 뿐이니까-그래서 `원형'이란 이름이 붙는다) 콘볼루션도 이 한정된 범위에서 수행한다고 생각하면 된다.
\paragraph{두 신호의 콘볼루션과 푸리에 급수 해석}
두 신호의 원형 콘볼루션에 대해 푸리에 급수 해석을 하면 다음과 같다.
\begin{IEEEeqnarray*}{rCl}
    \mathcal{F}(f[n] \circledast g[n])&=&\mathcal{F}\left(\sum_{m=0}^{N-1}f[m] g[n-m]\right)\\
    &=&\sum_{m=0}^{N-1}f[m]\mathcal{F}(g[n-m])\\
    &=&\sum_{m=0}^{N-1}f[m]\mathcal{F}(g[n])e^{-j2\pi F_0 k m}\\
    &=&\mathcal{F}(g[n])\sum_{m=0}^{N-1}f[m]e^{-j2\pi F_0 k m}\\
    &=&\mathcal{F}(g[n])\mathcal{F}(f[n])=\mathcal{F}(f[n])\mathcal{F}(g[n])\IEEEyesnumber
\end{IEEEeqnarray*}
\subsection{Parseval의 정리}
Parseval의 정리는 다음과 같다.
이산 시간 주기 신호의 전력은 푸리에 급수의 계수의 크기 제곱들을 다 더한 것과 같다는 의미이며, 각 푸리에 급수의 크기의 제곱은 각 주파수 성분이 갖는 전력임을 뜻한다.
\begin{equation}
    \frac{1}{N}\sum_{n=0}^{N-1}(\left\vert f[n]\right\vert)^2=\sum_{k=0}^{N-1}(\left\vert c_k\right\vert)^2
\end{equation}


\section{이산 시간 비주기 신호의 주파수 해석}