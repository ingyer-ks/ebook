\chapter{이 문서에 대해}

\section{문서 작성 경위와 목표}
올해(2020년) 상반기, 늘 그랬듯이 뭘 지를까 고민하던 나는 쓸데없이 NAS가 사고 싶었다.
그래서 로켓처럼 빠른 곳에서 NAS 본체와 하드디스크들을 주문했다.
그러고 나니 지름을 합리화할 무언가가 필요해서 블로그를 시작했다. 처음엔 특별한 목적이 있는 것은 아니었다.
그냥 서버 만들기와 관리하기 연습도 해볼 겸 시작했다.
그러다, 코로나-19가 퍼지기 시작하자 집에 있는 시간이 많아졌다.
심심했던 나는 뭘 할까 고민하다가, 각종 통신직 공무원/군무원/공기업 취업 방에 들어갔다.
거기서 놀다 보니 어느날은 블로그에 전자공학, 그 중에서도 통신 관련 내용과 문제풀이를 써보면 어떨까 하는 생각이 들었다.
그래서 \href{https://blog.ingyerlog.kr}{ingyerlog-잉여롭게} 블로그를 시작하게 되었다.
그러다 한발 더 나아가서 \href{https://youtube.ingyerlog.kr}{유튜브}도 잠깐 하게 되었다.
\par
이런 일들을 하면서 든 생각은, 내가 학교에서 공부할 때 사용한 책들은 너무 방대하고, 수험서들은 설명이 빈약하다는 것이었다.
그 중간 정도에서, 통신직 시험을 위해 공부할 때 도움이 될 법한 책은 없을까 하는 생각이 들었다.
물론 있긴 하겠지만, 언젠가는 내가 한번 뭔가 만들어보면 어떨까 하는 생각도 하게 되었다.
그래서 gitbook을 이용해볼까도 하고, 워드프로세서로 문서 작성 시작을 해볼까 고민도 했었다.(고민만)
그러던 어느 날, 2020년이 거의 다 간 지금 12월 중순 즈음, 이전에 몇 번 들어만 보았던 \LaTeX에 불현듯 꽂히게 되었다.
업무 문서도 \LaTeX로 만들어볼까 싶기도 하였고, 또 내가 생각해오던 걸 하는 데 바로 이것이 적합하지 않으면 무엇이 적합하랴 싶었다.
그래서, 책도 질렀고, 시작하려면 지금 시작해야겠단 생각이 들어서, 오늘-이 아니라 어제, 12시가 어느새 넘었다-이렇게 글을 쓰기 시작했다.
다 쓰는 데에 얼마나 걸리려나...
\par
위에서 밝혔듯이, 학교 전공 과목 교재보다는 가볍게, 문제풀이용 수험서보다는 자세하게 내용을 다뤄보고자 한다.
학교 수업에서도 두꺼운 전공책 전체를 다루는 일은 생각보다 적지 않은가?
그렇다고 수험용 책을 보자니 설명이 부족한 것을 몇 번 보았다.
그 중간 지점에서 편한 마음으로 읽어갈 수 있는 문서를 만들고자 한다.

\section{추천 도서 및 문서}
더 자세하고 정확한 내용을 원한다면 아래 문서와 책들을 참고하면 좋겠다.
\begin{itemize}
    \item Haykin \& Moher. (2007). Haykin의 통신이론 (최형진, 공형윤, 김기도, 김대진, 김철성, 이태홍, 장태규 역). 서울:한티미디어. (원서출판 2006) \\
    - 그리 두껍지 않으면서 확률까지 다루는 알찬 책이다.
    \item Richard Baraniuk. (2015). \href{https://cnx.org/contents/77608400-65b9-4f03-8a5f-536c611866bb@15.4/Signals-and-Systems}{Signals and Systems}. \\
    - 기초 신호와 시스템 장을 쓸 때 많은 도움을 받았다. 거의 번역했다고 봐도 될 것이다. \href{https://creativecommons.org/licenses/by/4.0/deed.ko}{크리에이티브 커먼즈 라이센스 저작자표시 4.0 국제(CC BY 4.0) 라이선스}를 따른다.
\end{itemize}

\section{피드백}
잘못된 내용을 발견하였거나 무엇이든 제안할 거리가 있다면 아래 경로로 알려주세요.
\begin{itemize}
    \item \href{mailto:ingyer.ks@gmail.com}{ingyer.ks@gmail.com}
    \item \href{https://github.com/ingyer-ks/Book}{github 레포지토리} (https://github.com/ingyer-ks/Book)
    \item \href{https://kakaotalk.ingyerlog.kr}{카카오톡 오픈채팅} (https://kakaotalk.ingyerlog.kr)
\end{itemize}
\section{도움을 주신 분들}
이 문서를 만드는 데 도움을 주신 분들이 많이 계신다.
전공책을 보겠냐 너가 쓴 글을 보겠냐라고 현실적인 조언을 해 주신 L모 사무관님(그건 형이 너무 잘나서 그런 거에요!!),
지식재산권과 관련하여 대충 보고 문제 없을 거 같다고 안심시켜준 J모 변리사 형, 좋은 글이 될 것 같다며 이런저런 조언을 해준, 곧 백수 생활 청산할 L형, 
남극에서도 글을 봐주고 계신(다고 믿는) N님, 초기 리뷰어가 되어 주신 공무원/군무원/공기업 오픈카톡방의 眞통신님, 유댕님, 안녕하세용님, 두두님 등이 계신다.
사촌동생 K선생, 엄마, 동생 등은 새로운 눈으로 글을 보아 주었다.
\href{http://www.ktword.co.kr/}{정보통신기술용어해설}의 차재복 선생님도 빼놓을 수 없다. 선생님이 계셔서 블로그도 시작하고, 이 글도 시작할 수 있었다. 늘 건강하셨음 좋겠다.
(정보통신기술용어해설에 내 블로그가 걸려있는 게 민망한 일이 되지 않아야 할 터인데...)
아무튼, 크고작은 도움을 주시는 모든 분들을 지속적으로 여기에 기록해나가고자 한다.

\section{지식재산권 관련}
주로 내가 학교에서 배우며 정리한 노트를 중심으로 내용을 채워갔으니 지식재산권 침해 문제는 없을 것이다.(없길 바란다 ㅠㅠ) 
그림들이나 그래프들은 내가 직접 그리거나, 크리에이티브 커먼즈 라이선스 등 사용 조건에 맞는 것들을 찾아서 사용하였다.
가끔 소개할 문제들은 \href{https://www.gosi.kr}{인사혁신처 사이버국가고시센터}와 \href{http://gosi.seoul.go.kr}{서울시인터넷원서접수센터}에 공개되어 있는 문제들이다.
인사혁신처 담당자로부터 비상업적 목적으로 사용하는 것은 문제가 없음을 구두로 확인받았다.
이 문서의 내용 중 내게 권리가 있는 콘텐츠에는 \href{https://creativecommons.org/licenses/by-nc/4.0/}{크리에이티브 커먼즈 저작자표시-비영리 4.0 국제 라이선스(CC BY-NC 4.0)}가 적용된다.
\\
\includegraphics{CC.png}