\chapter{MOSFET}
MOSFET은 Metal-Oxide-Semiconductor Field-Effect-Transistor의 줄임말이다. 릴리엔필드가 제안했고, 강대원 박사가 실현했다.
\section{MOSFET의 구조}
N타입 MOSFET(NMOS)의 구조는 \figurename~ \ref{fig:mos structure}\와 같다.
\begin{figure}
    \centering
    \image[height=4cm]{mos_structure.png}
    \caption{MOSFET의 구조}\label{fig:mos structure}
\end{figure}
가운데 초록색 부분은 금속이고, 좌우 n+영역은 각각 소스와 드레인이다.(단순한 MOS는 좌우 대칭형으로 만들 수 있기 때문에, 회로 구성에 따라 소스와 드레인이 서로 바뀔 수도 있다.)
소스란 이름은 여기서 캐리어가 공급되기 때문에 붙었고, 드레인이란 이름은 여기로 캐리어들이 하수구로 물 내려가듯이 빨려가기 때문에 붙었다.
이 n+영역은 기본적으로 P타입 기판에 도핑을 해서 만들어지는데, 이 P타입 기판을 substrate 혹은 바디라고 부른다.
위 MOSFET을 NMOS라고 부르는 이유는 사용되는 캐리어의 종류가 전자(negative)이기 때문이다.
N타입 MOSFET이 있으면 P타입 MOSFET(PMOS)도 존재한다. 그저 n과 p 타입을 바꾸면 된다. 당연히 사용되는 캐리어는 정공이 된다.
초록색 금속 아래의 파란 부분은 산화막으로, 전류의 흐름을 막는 역할을 한다. 이 산화막 때문에 DC전류가 흐르지 않고, 그래서 전력을 적게 쓰는 것이다.

\section{채널}
채널이란 캐리어가 이동할 수 있는 통로를 말한다. 게이트에 전압을 걸어서 바디에 있는 캐리어들을 뒤쪽으로 밀어내면 그 곳에는 도너나 억셉터 이온들만 남게 된다. 즉 공핍 영역이 된다(\figurename~\ref{fig:weak inversion}).
게이트에 전압을 더 크게 걸게 되면 소수 캐리어들이 그 극성에 이끌려서 산화막 아래에 고이게 된다(\figurename~\ref{fig:strong inversion}). 이 캐리어가 고이는 영역이 바로 채널이다. 두 그림은 모두 NMOS 기준이다.
\begin{figure}
    \centering
    \image[height=4cm]{weak_inv.png}
    \caption{캐리어가 밀려나고 산화막 아래에 억셉터 이온만 남은 상태}\label{fig:weak inversion}
    \image[height=4cm]{strong_inv.png}
    \caption{전자까지 끌려온 상태}\label{fig:strong inversion}
\end{figure}

\section{전압과 전류의 관계}
MOS는 3극 소자이므로, 소스 단을 기준으로 하면 전압은 게이트-소스 간, 드레인-소스 간 두 개가 존재한다. 따라서 각 전압과 전류의 관계를 분석하기 위해서는 하나를 고정하고 다른 하나를 바꾸면서 생각해보아야 한다.
\subsection{게이트 전압과 전류의 관계}
NMOS에 대해서 생각하자. 같은 드레인 전압에 대해서 게이트 전압이 전류에 끼치는 영향이 어떤지 조사할 것이다.
게이트 전압이 어느 정도 지점까지는 P타입 바디에서 양공을 밀어내는 데 쓰이므로 채널이 형성되지 않는다. 그러다가 특정 전압을 넘어서면 그 때부터는 전자가 고이기 시작하고, 그 이후로 가해지는 전압은 전자를 당기는 데에만 쓰인다.
이 전압을 문턱 전압(Threshold voltage, $V_{th}$)라고 부른다. 문턱 전압보다 전압이 크다면, 캐리어가 다니는 길이 위아래로 넓어지는 효과가 나므로 저항이 줄어드는 것과 마찬가지이다.(\figurename~\ref{fig:vg effect on id}) 따라서 전류량은 커지게 된다.
\begin{figure}[!hpb]
    \centering
    \image[height=8cm]{vg_effect_on_id.png}
    \caption{게이트 전압이 전류에 끼치는 영향}\label{fig:vg effect on id}
\end{figure}
\subsection{산화막과 전류 관계}
산화막이 길다면 캐리어가 이동하는 길이가 늘어나므로 저항이 길어지는 것과 마찬가지이다. 따라서 산화막이 길어지면 전류는 줄어든다.
산화막의 두께도 영향을 미친다. 산화막이 두껍다면 전압을 걸어도 그로부터 형성되는 전계가 약해져서 효과적으로 다수 캐리어를 밀어내기 어려울 것이다. 그에 따라 같은 전류를 얻기 위해서는 더 많은 전압을 가해야만 한다.
산화막이 넓다면 저항의 단면적이 넓어지는 효과가 발생하므로 전류량은 늘어난다.
\subsection{핀치 오프}
지금까지는 드레인 전압을 딱히 생각하지 않았다. 이제 NMOS에서 드레인의 전압을 올려보자. 그러면 전압 분포는 대충 \figurename~ \ref{fig:ch potential variation}\과 같이 될 것이다.
\begin{figure}[!hbp]
    \centering
    \image{channel_potential.png}
    \caption{게이트 전압과 드레인 전압이 존재할 때 채널 각 위치에서의 전위차}\label{fig:ch potential variation}
\end{figure}
만약, $V_D-V_G=V_{th}$가 될 때까지 드레인 전압이 커진다면, 채널의 드레인 쪽 끝은 채널 두께가 0이 되어 버린다. 이러면 전류가 못 흐르게 되지 않을까?하고 생각할 수 있으나, 그렇지가 않다.
다이오드에서 역방향 바이어스를 걸었던 상황을 기억하는가? N타입쪽에 P타입보다 높은 전압이 걸리게 되면 N에서 P로 드리프트 전류가 흘렀다. 그리고 이 때 캐리어 이동 속도가 충분히 빠르다고 했고, 그에 따라 전류량은 결국 캐리어 농도에만 의존한다고 했다.
여기서도 마찬가지 상황이 발생한다. P타입 바디와 N타입 드레인 사이에 역전압이 걸리고, 이 공핍층쪽으로 들어온 전자는 강한 전기장에 의해 빠르게 이동한다. 결국 드레인 전압보다는 채널로부터 바디-드레인 간의 공핍층에 얼마나 자주 전자가 들어오느냐가 전자를 지배하게 되는 것이다.
정리하면 드레인 전압이 어느 정도까지 커질 때까지는 저항에 전압이 많이 걸리는 것과 같은 효과가 나지만, 어느 시점부터는 점점 드레인 전압이 커지더라도 전류가 커지지 않게 되고, 결국 전류는 일정하게, 즉 포화되게 한다.
그래서 이렇게 전류가 일정하게 되는 영역의 이름을 포화 영역이라 한다.(나중에 나올 BJT의 포화 영역과 의미가 다르다) 그래프로 그리면 \figurename~\ref{fig:vds vs id curve}\와 같다.
\begin{figure}[!hbp]
    \centering
    \image{mos_sat_current.png}
    \caption{드레인 전압과 전류 사이의 관계}\label{fig:vds vs id curve}
\end{figure}

\subsection{전류의 정량적 분석}
MOSFET의 전류는 산화막에 의한 커패시터로 인해 쌓이는 전하가 소스-드레인 간 전압에 의해 움직이는 것이라고 볼 수 있다. 따라서 먼저 게이트 산화막에 걸리는 전압에 의해 전하가 얼마나 쌓이는지를 분석해봐야 한다.
먼저 드레인에 전압이 걸리지 않을 때인 다음 식을 보자.
\begin{equation}
    Q=WC_{ox}(V_{GS}-V_{th})
\end{equation}
위 식에서 Q는 채널에 쌓이는 전하이고, W는 산화막의 너비이다. $C_{ox}$는 산화막의 단위너비당 커패시턴스이다. $Q=CV$와 유사하지만 전압이 $V_{GS}-V_{th}$로 주어지는 것에 유의하자.
문턱전압과 게이트 전압의 차이만이 전류에 기여하는 캐리어 전하를 끌어오게 되기 때문에 이렇게 된다.
다음으로, 드레인에 전압이 걸리는 경우를 생각해보자. 각 위치 $x$에서의 전압을 $V(x)$라 하면 산화막 양단에 걸리면서 문턱 전압을 제외한 전압은 $V_{GS}-V(x)-V_{th}$이므로, 각 위치 $x$에서의 미소전하량은 다음과 같이 될 것이다.
\begin{equation}
    Q(x)=WC_{ox}[V_{GS}-V(x)-V_{th}]
\end{equation}
다음으로, 전류란 것은 어떤 전하들이 단위시간당 움직이는 양이므로, 각 미소전하 $Q(x)$들이 각 위치에서 받는 전기장에 의한 속도를 곱한 것이 각 위치에서의 전류가 될 것이다.
\begin{IEEEeqnarray*}{rCl}
    I_D&=&Q(x)v(x)\\
    &=&WC_{ox}[V_{GS}-V(x)-V_{th}] \mu_n\frac{dV(x)}{dx}\\
    \int_{0}^{L} I_D dx&=&\int_{0}^{V_{DS}}WC_{ox}[V_{GS}-V(x)-V_{th}] \mu_ndV\\
    I_D L&=&\mu_nC_{ox}W((V_{GS}-V_{th})V_{DS}-(1/2)V^2_{DS})\\
    I_D&=&\frac{1}{2}\mu_nC_{ox}\frac{W}{L}(2(V_{GS}-V_{th})V_{DS}-V^2_{DS})\IEEEyesnumber
\end{IEEEeqnarray*}
그런데, 이 이차함수는 위로 볼록하므로 꼭지점인 $V_{DS}=V_{GS}-V_{th}$를 지나게 되면 전류가 감소한다. 근데 그럴 리가 있는가? 전압이 커지는데 전류가 늘진 못할망정 줄어드는 건 말이 안 된다. 따라서 이 지점 이후로는 그냥 이 전류가 유지될 것이다.
따라서 이 지점 이후로 전류는 다음과 같이 된다.
\begin{equation}
    I_D=\frac{1}{2}\mu_nC_{ox}\frac{W}{L}(V_{GS}-V_{th})^2
\end{equation}
그래프를 그려보면 \figurename~\ref{fig:mos sat current curve}\와 같다.
\begin{figure}[!tbp]
    \centering
    \image{mos_sat_current_curve.pdf}
    \caption{$V_{DS}$-$I_D$ 곡선}\label{fig:mos sat current curve}
\end{figure}

\subsection{MOS의 출력 저항}
앞서 식 \ref{eqn:dynamic resistance}에서 다이내믹 저항을 구하는 법을 알아보았다. MOS의 경우, 이상적이라면 드레인 쪽에서 바라본 출력 저항은 무한대여야 한다(=\figurename~\ref{fig:mos sat current curve}의 포화 영역에서 기울기가 평평해야 한다).
하지만, 드레인 전압이 올라가면 핀치 오프에 의해 채널 길이가 짧아지는 효과가 나타나고, 이는 저항의 길이가 짧아지는 효과가 난다. 즉 저항이 무한대보다 작아지고, 드레인 전압이 올라가면 전류도 증가하게 된다.
이를 채널 길이 변조(Channel Length Modulation)라 하고, $r_o$로 표기한다.

\subsection{전달 컨덕턴스(Transconductance)}\label{subsection:mos transconductance}
전달 컨덕턴스는 게이트-소스 간에 작은 교류 전압을 걸었을 때 드레인에 얼마나 전류가 흐르는지 나타내는 비율이다.
일반적인 저항은 전압을 거는 단자와 전류가 흐르는 단자가 같지만, MOS에서는 둘이 서로 다르다. 따라서 전압 신호를 전달하여서 전류가 흐르게 한다는 의미로 전달 컨덕턴스라는 용어를 사용한다.
포화 영역에서 전달 컨덕턴스는 수식으로 쓰면 다음과 같다.
\begin{IEEEeqnarray*}{rCl}
    I_D&=&\frac{1}{2}\mu_nC_{ox}\frac{W}{L}(V_{GS}-V_{th})^2\\
    g_m&=&\frac{\partial I_D}{\partial V_D}\\
    &=&\mu_nC_{ox}\frac{W}{L}(V_{GS}-V_{th})\IEEEyesnumber\label{eqn:mos transconductance}
\end{IEEEeqnarray*}
이 식은 위 계수들 중에 어떤 것이 변하고 어떤 것이 고정이냐에 따라 다양하게 표현된다. 직접 해보시라.
위 식의 의미를 그래프로 그리면 다음과 같다.
\begin{figure}[!hbp]
    \centering
    \image[height=8cm]{mos_transconductance.pdf}
    \caption{실제 $I_D-V_G$ 곡선과 전달 컨덕턴스로 선형화된 $I_D-V_G$ 직선}
\end{figure}

위 그래프에서 특정 DC 지점에서의 전달 컨덕턴스(기울기)를 이용하여 접선을 그렸다. 이 DC 지점을 바이어스 포인트(동작점)이라고 한다. 당연히 전달 컨덕턴스는 동작점이 어디냐에 따라 달라질 것이다.
동작점 근처에서 원래 곡선과 접선간의 차이를 무시할 수 있을 만큼 작은 교류 전압을 가하면 그 교류 출력은 입력 교류 전압과 전달 컨덕턴스의 곱으로 나타난다.
결국 이는 수학에서 말하는 테일러 1차 근사(선형 근사)일 뿐이다. 이렇게 근사를 하면 선형성에 의해서 문제를 쉽게 풀 수 있기 때문에 작은 신호에 대해서는 원래의 이차함수를 사용하지 않아도 된다.

\subsection{소신호 해석}
소신호 해석이란 MOS에 가해지는 신호가 DC 성분과 작은 AC 성분으로 나눠진다면, 먼저 DC 성분에 대해서 전압, 전류, 전달 컨덕턴스 등을 구한 후 이들을 바탕으로 교류 입력 신호에 대해서 출력을 해석하는 것을 말한다.
이를 위해서는 MOS를 아래와 같이 등가 소신호 모델로 바꿔야 한다.
\paragraph{$\pi$ 모델}
$\pi$ 모델이란 등가 회로의 모양이 $\pi$ 형태라서 붙은 모델을 말한다. 다음 \figurename~\ref{fig:mos pi small signal model}\을 보자.
\begin{figure}[!hpb]
    \centering
    \image{mos_pi_small_signal_model.PNG}
    \caption{MOSFET의 $\pi$ 소신호 모델}\label{fig:mos pi small signal model}
\end{figure}
\paragraph{T 모델}
T 모델은 이와 달리 형태가 T처럼 생긴 모델을 말한다. 다음 \figurename~\ref{fig:mos t small signal model}\을 보자.
\begin{figure}[!tpb]
    \centering
    \image[width=8cm]{mos_t_small_signal_model.PNG}
    \caption{MOSFET의 T 소신호 모델}\label{fig:mos t small signal model}
\end{figure}
$\pi$ 모델에선 끊어져 있던 게이트가 연결되어 있어서 전류가 흐를 것 같지만 실제로 그렇진 않을 것이다. 왜 그럴까?

\section{PMOS}
PMOS의 경우는 지금까지 나온 전압들 간의 관계를 뒤집으면 된다. 소스보다 게이트 전압이 낮아야 작동하고, 게이트 전압보다 드레인 전압이 어느 정도 이상 내려가게 되면 포화 영역에서 작동하게 된다.
