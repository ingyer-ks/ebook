\chapter{MOSFET 회로}
앞서 배운 MOSFET을 이용한 회로들을 해석하는 방법을 알아보자.

\section{임피던스}
MOSFET의 각 단자의 임피던스를 알아보자. 관심있는 단에 테스트 전압을 걸어서 흘러 들어가는 전류의 비율을 구할 것이다. 그리고 다른 단은 모두 접지에 연결해서 영향이 없게 해야 한다.
\subsection{게이트의 입력 임피던스}
\figurename~\ref{fig:mos pi small signal model}\을 보자. 게이트는 어디에도 연결되어 있지 않다. 따라서 들어가는 전류가 0이니 임피던스는 무한대이다.
한편, \figurename~\ref{fig:mos t small signal model}에선 어떨까? 게이트가 다른 곳과 연결되어 있는 것 같다. 하지만 역시 들어가는 전류는 0이다. 왜 그럴지 생각해보자.
\subsection{드레인의 출력 임피던스}
드레인은 출력을 뽑는 단이기 때문에 출력 임피던스라 한다. 게이트와 소스를 접지해보자.
\begin{figure}[!tpb]
    \centering
    \image{drain_output_impedance.PNG}
    \caption{드레인 출력 임피던스}\label{fig:drain output impedance}
\end{figure}
\figurename~\ref{fig:drain output impedance}에서 $V_{GS}=0-0=0$이므로 $g_mV_{GS}=0$이다. 따라서 남는 것은 $r_o$이므로 드레인 출력 임피던스는 $r_o$이다.
\subsection{소스의 입력 임피던스}
소스 단에서 들여다본 임피던스를 구해 보자. 드레인을 접지하고 소스에 테스트 전압을 걸어 보자.
\begin{figure}[!hpb]
    \centering
    \image[width=10cm]{src_input_impedance.PNG}
    \caption{소스 입력 임피던스}\label{fig:src input impedance}
\end{figure}
\figurename~\ref{fig:src input impedance}에서 $V_{GS}=0-V_{test}=-V_{test}$이다. 따라서 종속 전류원은 위쪽으로 $g_mV_{test}$만큼의 전류를 흘린다.
이와 병렬로 드레인 출력 임피던스 $r_o$가 있으므로 이를 통해서 $V_{test}/r_o$가 흐른다.
따라서 흘러 들어가는 총 전류는 $g_mV_{test}+V_{test}/r_o=V_{test}(g_m+1/r_o)$이다.
그러므로 총 임피던스는 $\frac{V_{test}}{V_{test}(g_m+1/r_o)}=\frac{1}{(g_m+1/r_o)}=\frac{\frac{r_o}{g_m}}{r_o+1/g_m}=\left. r_o\left.\right\vert\right\vert\frac{1}{g_m}$이다.

\section{증폭기}
MOS를 활용하면 입력 교류 전압보다 큰 출력 교류 전압을 만들어낼 수 있다. 그래서 증폭기라고 한다. (작게도 만들 수 있다.)
가능한 세 가지 증폭기 회로 구성을 알아보자.

\subsection{공통 소스 구성}
공통 소스 구성이란 게이트로 신호를 입력하고 드레인에서 출력을 뽑는 구성이다. 소스는 게이트와 드레인의 공통 기준 전압 역할을 하기 때문에 공통 소스 구성이라고 한다. \figurename~\ref{fig:cs stage}\을 보자.
\begin{figure}[!hpb]
    \centering
    \image[width=7cm]{cs_topology.PNG}
    \caption{일반적인 공통 소스 회로}\label{fig:cs stage}
\end{figure}
\paragraph{전압 이득}
이 회로에서 $V_{GS}$는 얼마일까? $V_{in}$이 게이트와 $R_S$에 나눠 걸린다. 소신호 회로로 바꿔서 생각해보자(\figurename~\ref{fig:cs small signal}). 소신호는 교류만을 다루므로 DC 전압, 전류원은 0으로 처리한다. 또한 $r_o$는 매우 크다고 가정하자.
\begin{figure}[!hpb]
    \centering
    \image[width=6cm]{cs_small_signal.PNG}
    \caption{공통 소스 소신호 회로}\label{fig:cs small signal}
\end{figure}
\figurename~\ref{fig:cs small signal}에서 $V_{GS}=V_{in}\frac{1/g_m}{1/g_m+R_S}$임을 알 수 있다. 이 전압에 의해서 $g_mV_{gs}$전류원이 전류를 제일 위의 AC 그라운드에서 $R_D$를 통해서 전류를 끌어온다.
그리고 그로 인한 전압 강하인 $0-g_mV_{GS}R_D=-g_mV_{GS}R_D$가 $V_{out}$이 된다.
전압이득을 $A_v$라 하고 정리하면 다음과 같다.
\begin{IEEEeqnarray*}{rCl}
    A_v&=&\frac{V_{out}}{V_{in}}=\frac{-g_mV_{in}\frac{1/g_m}{1/g_m+R_S}R_D}{V_{in}}\\
    &=&-\frac{R_D}{1/g_m+R_S}\IEEEyesnumber
\end{IEEEeqnarray*}
이다.

이 수식을 다른 관점에서 해석해보자. 어차피 $R_D, 1/g_m, R_S$에는 모두 같은 어떤 전류 $I$가 흐른다. \figurename~\ref{fig:cs small signal}에서 $V_{in}$은 $(1/g_m+R_S)I$이다.
한편 출력 전압은 $V_{out}-0=V_{out}=-R_DI$이다. 따라서 전압이득은 $A_v=\frac{-R_DI}{(1/g_m+R_S)I}=\frac{-R_D}{1/g_m+R_S}$임을 바로 알 수 있다.
이를 풀어서 쓰면 다음과 같다.
\begin{equation}
    (전압 이득)=-\frac{위쪽 임피던스}{아래쪽 임피던스}
\end{equation}

\paragraph{입출력 임피던스}
이 회로의 입력 임피던스는 게이트 임피던스이므로 무한대이고, 출력 임피던스는 $r_o$를 무시하면 $R_D$이다. 무시하지 않으면 어려워지므로 넘어가겠다.
입력 임피던스가 큼으로 생기는 장점은 무엇일까?
\begin{figure}
    \centering
    \image[height=6cm]{signal_to_cs_circuit.png}
    \caption{외부 신호원에 연결된 공통 소스 회로}\label{fig:cs with external voltage signal}
\end{figure}
\figurename~\ref{fig:cs with external voltage signal}에서 외부 입력 신호원에 어떤 저항 $R_{sig}$가 있다고 하자. 이게 일반적인 경우일 것이다.
여기서 공통 소스 회로의 입력 임피던스 $R_{in}$이 크다면 신호원의 신호 $V_{sig}$는 $R_{in}$에 많이 분배될 것이다. 따라서 전압을 감지하려면 입력 임피던스가 큰 것이 유리하다.
이것이 연산 증폭기 같은 전압 감지 회로에서 입력 임피던스가 커야 하는 이유이다.
\paragraph{소스 디제너레이션 저항과 바이패서 커패시터}
그런데, $R_S$는 왜 있는 것일까? 이 때문에 전압 이득이 작아지지 않는가? 그 이유는 $R_S$가 네거티브 피드백을 함으로써 안정성을 좋게 하기 때문이다.
가령, DC로서 일정해야 할 바이어스 포인트가 흔들린다고 해보자. 그럼 출력 또한 흔들리게 될 것이다.
그런데 $R_S$가 존재하면 입력 바이어스 포인트의 흔들림이 분배되어서 $V_{GS}$의 흔들림은 입력의 흔들림보다는 작아진다. 따라서 더 안정적인 동작을 할 수 있다.
그렇다면, 이 작아지는 전압 이득을 보상할 방법은 없는가? \figurename~\ref{fig:cs with bypass cap}\을 보자.
\begin{figure}[!tpb]
    \centering
    \image[width=7cm]{cs_with_bypass_cap.PNG}
    \caption{바이패스 커패시터가 있는 공통 소스 회로}\label{fig:cs with bypass cap}
\end{figure}
$R_S$ 옆에 $C_S$를 달았다. 커패시터가 충분히 커서 우리가 원하는 주파수 $\omega$에 대해 커패시터의 임피던스 $\frac{1}{j\omega C_S}$가 충분히 작다면, 커패시터는 그라운드와 소스를 도통시키는 도선 역할을 하게 된다.
따라서 교류 영역에서는 $R_S$가 없는 것과 마찬가지가 되어서 전압 이득이 $R_S$가 없는 $g_mR_D$로 커지게 된다.

\subsection{공통 게이트 구성}
공통 게이트 구성이란 게이트 단이 소스와 드레인의 공통 기준 전압이 되고, 소스로 전압 신호를 입력하는 구성을 말한다. \figurename~\ref{fig:cg topology}\를 보자.
\begin{figure}[!hpb]
    \centering
    \image[width=8cm]{cg_topology.PNG}
    \caption{공통 게이트 구성}\label{fig:cg topology}
\end{figure}
\paragraph{전압 이득}
이 회로에서 $V_{GS}$는 $-V_{in}$이므로 종속전류원 $g_mV_{GS}$는 위쪽으로 $g_mV_{in}$의 전류를 흘릴 것이다. 따라서 $V_{out}$의 전압은 $g_mV_{in}R_D-0=g_mV_{in}R_D$이 되므로 전압 이득은 최종적으로 다음과 같다.
\begin{equation}
    A_v=\frac{V_{out}}{V_{in}}=\frac{g_mV_{in}R_D}{V_{in}}=g_mR_D
\end{equation}
이 식을 보면 공통 소스 구성과 크기는 같고 위상(부호)이 반대임을 알 수 있다. 이는 당연한 결과로, 그저 $V_{GS}$가 반대가 되니까 그런 것이다.
\paragraph{입출력 임피던스}
입출력 임피던스를 살펴보자. 먼저 출력 임피던스는 $R_D$로 공통 소스 구성과 동일할 것이다. 입력 임피던스는 어떨까? 공통 소스 구성에서 입력 임피던스는 무한대였다. 공통 게이트 구성에서는 어떻게 될지 \figurename~\ref{fig:cg input impedance}를 보며 생각해보자.
\begin{figure}[!tbp]
    \centering
    \image[width=7cm]{cg_input_impedance.PNG}
    \caption{공통 게이트 구성의 입력 임피던스를 구하기 위한 회로}\label{fig:cg input impedance}
\end{figure}
$V_{test}$가 공급하는 전류는 얼마일까? 바로 종속 전류원이 위쪽으로 흘리게 되는 전류인 $g_m(-V_{GS})=-g_m(0-V_{test})=g_mV_{test}$이다. 따라서 입력 임피던스는 $V_{test}/(g_mV_{test})=1/g_m$이다.
이 값은 보통 작다. 실제로 수십 옴 정도였던 것 같다. 왜 이렇게 작은 입력 임피던스가 필요할까?
전기자기학에서 전송선로 쪽을 공부해본 적이 있다면 알 것이다. 보통 전송선로의 임피던스는 $50 \Omega $ 혹은 $70 \Omega$이다.
따라서 이러한 전송선로에 임피던스 매칭을 위해서는(임피던스 매칭을 하지 않으면 반사파가 생겨서 신호 전달이 잘 되지 않는다) 끝에 작은 임피던스가 달려야 한다. 이럴 경우, 공통 소스 구성보다는 공통 게이트 구성이 유리한 것이다.

\subsection{소스 폴로워 구성}
마지막 구성인 소스 폴로워 구성을 살펴보자. \figurename~\ref{fig:src follower topology}\를 보자.
\begin{figure}[!hbp]
    \centering
    \image[height=7cm]{src_follower.PNG}
    \caption{소스 폴로워 구성}\label{fig:src follower topology}
\end{figure}
\paragraph{전압 이득}
이 회로의 전압 이득은 어떻게 될까? T 모델을 적용해서 생각해보자.
\begin{figure}[!hbp]
    \centering
    \image[height=7cm]{src_follower_eq_circuit.PNG}
    \caption{소스 폴로워의 T 모델 소신호 등가회로}\label{fig:src follower eq circuit}
\end{figure}
\figurename~\ref{fig:src follower eq circuit}에서 $V_{out}$은 어떻게 구할 수 있을까? 바로 $V_{in}$이 $1/g_m$과 $R_S$에 분배될 때 $R_S$가 가져가는 전압이 $V_{out}$임을 알 수 있다.
따라서 전압 이득 $A_v$을 수식으로 쓰면 다음과 같다.
\begin{equation}
    A_v=\frac{V_{out}}{V_{in}}=\frac{\frac{R_S}{1/g_m+R_S}V_{in}}{V_{in}}=\frac{R_S}{1/g_m+R_S}
\end{equation}
이 값은 증폭기란 이름이 어울리지 않게도 1보다 항상 작다. $R_S$가 커지면 1에 가까이 갈 뿐이다(다만 부호가 +라서 입력과 출력 위상이 동일하기 때문에 소스를 따라간단 의미에서 소스 폴로워란 이름이 붙었다).
그럼 왜 이런 회로를 쓰는 걸까?
\paragraph{입출력 임피던스}
입출력 임피던스를 생각해보자. 입력 임피던스는 게이트에 전류가 흐르지 않음에 따라 무한대이다. 출력 임피던스는 \figurename~\ref{fig:src follower eq circuit}에서 볼 수 있는 것처럼 위쪽으로 $1/g_m$, 아래쪽으로 $R_S$가 있으므로 이 둘의 병렬인데, 보텅 $1/g_m$이 훨씬 작으므로 출력 임피던스는 $1/g_m$에 가까운, 그러나 더 작은 값이다.
이 작은 출력 임피던스는 $V_{out}$을 받아가는 부하에 전압 신호를 전달하기 유리하다. 다음 그림을 보자.
\begin{figure}[!hbp]
    \centering
    \image[height=6cm]{src_follower_with_load.PNG}
    \caption{부하 저항이 달린 소스 폴로워 테브닌 등가 회로}\label{fig:src follower Thevnin eq circuit}
\end{figure}
\figurename~\ref{fig:src follower Thevnin eq circuit}에서 소스 폴로워의 출력 전압 $V_{out}$이 부하 $R_L$에 잘 전달되려면 $R_{out}$은 어때야 할까? 0에 가까우면 가까울수록 $R_{out}$의 전압 강하가 작아져서 부하에 더 전압이 잘 전달될 것이다.
만약 공통 소스 구성이었다면 어땠을까? 출력 임피던스가 크므로 출력 전압의 상당 부분이 자체 출력 임피던스에서 강하가 일어나버릴 것이다.
이것이 연산 증폭기 같은 전압 출력 회로에서 출력 임피던스가 작아야 하는 이유이다.

\subsection{중간 정리}
위 내용을 정리하면 다음과 같다.
\begin{itemize}
    \item 공통 게이트 회로: 입력 임피던스 작음, 출력 임피던스 큼, 입출력 전압 위상 같음, 전압 이득 큼
    \item 공통 소스 회로: 입력 임피던스 큼, 출력 임피던스 큼, 입출력 전압 위상 반대, 전압 이득 큼
    \item 소스 폴로워 회로: 입력 임피던스 큼, 출력 임피던스 작음, 입출력 전압 위상 같음, 전압 이득 작음
\end{itemize}
\subsection{캐스코드와 다이오드형 연결 형태}

\subsection{바이어싱과 AC 커플링}

\subsection{캐스코드}

\subsection{주파수 응답}