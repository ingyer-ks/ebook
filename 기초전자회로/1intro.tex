\chapter{이 문서에 대해}

\section{문서 작성 경위와 목표}
기초 통신 이론만 쓰다 보니 다른 것도 잠깐씩 해보고 싶었다. 결국 전자기학, 전자회로/회로이론, 통신을 모두 다루는 게 목표이니까.
전자회로는 사실 어렵지 않다. 라자비나 세드라 같은 두꺼운 책의 절반도 안 되는 분량만 공부하면, 내 생각에 7급 문제는 다 풀 수 있다.
최근 5개년 정도의 기출문제를 분석해 보았을 때 내린 결론이다.
따라서 원래 이 시리즈를 쓰는 목적에 맞게 실용적인 수험에 도움 되는 글이 되고자 하여 소자들의 기본 원리를 간단히 다루고, 응용 회로에 대해서 간단히만 분석해볼 것이다.
어려운 피드백, 주파수 응답은 제외할 생각이다. 나부터 잘 모르니까.
그리고 미리 고지하지만 필자는 반도체 잘 모른다. 그래서 원리에 틀린 게 많을 터이니 이 글을 보시는 분들은 주의 바란다.

\section{추천 도서 및 문서}
더 자세하고 정확한 내용을 원한다면 아래 문서와 책들을 참고하면 좋겠다.
\begin{itemize}
    \item Floyd. (2019). 전자회로: Electronic Devices (손상희 , 강문상 , 김남훈 , 김민회 , 류근관 , 정연호 , 채용웅 , 최채형 , 황석승 역). 퍼스트북. (원서출판 2018) \\
    - 이 글의 목적에 가장 가까운 교재이다. 난 본 적이 없는 책이지만 쉽게 설명되어 있다고 들었다. 또한 라자비, 세드라에는 없는 전원회로나 사이리스터 등의 다른 소자도 다루고 있다. 가장 7급에 적합한 책이라 생각한다.
    \item Razavi. (2015). 마이크로전자회로 (김철우 , 김남수 , 김종선 , 박상규 , 백흥기 , 이강윤 , 정성욱 옮김). 한티미디어. (원서출판 2012)\\
    - 내가 학교에서 공부할 때 교재이다. 세드라 대비 적절한 두께와 설명이라고 생각되지만, 그래도 7급 시험용으로는 불필요한 부분이 많다.
\end{itemize}

\section{피드백}
잘못된 내용을 발견하였거나 무엇이든 제안할 거리가 있다면 아래 경로로 알려주세요.
\begin{itemize}
    \item \href{mailto:ingyer.ks@gmail.com}{ingyer.ks@gmail.com}
    \item \href{https://github.com/ingyer-ks/Book}{github 레포지토리} (https://github.com/ingyer-ks/Book)
\end{itemize}
\section{도움을 주신 분들}
크고작은 도움을 주시는 모든 분들을 지속적으로 여기에 기록해나가고자 한다.

\section{지식재산권 관련}
주로 내가 학교에서 배우며 정리한 노트를 중심으로 내용을 채워갔으니 지식재산권 침해 문제는 없을 것이다.(없길 바란다 ㅠㅠ) 
그림들이나 그래프들은 내가 직접 그리거나, 크리에이티브 커먼즈 라이선스 등 사용 조건에 맞는 것들을 찾아서 사용하였다.
가끔 소개할 문제들은 \href{https://www.gosi.kr}{인사혁신처 사이버국가고시센터}와 \href{http://gosi.seoul.go.kr}{서울시인터넷원서접수센터}에 공개되어 있는 문제들이다.
인사혁신처 담당자로부터 비상업적 목적으로 사용하는 것은 문제가 없음을 구두로 확인받았다.
이 문서의 내용 중 내게 권리가 있는 콘텐츠에는 \href{https://creativecommons.org/licenses/by-nc/4.0/}{크리에이티브 커먼즈 저작자표시-비영리 4.0 국제 라이선스(CC BY-NC 4.0)}가 적용된다.
\\
\includegraphics{CC.png}