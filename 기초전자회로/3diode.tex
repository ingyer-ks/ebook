\chapter{다이오드}
다이오드란 P타입 반도체와 N타입 반도체를 붙여서 PN 접합을 만든 소자를 의미한다. 다이오드의 원리를 살펴보자.
\section{평형 상태의 다이오드}
P타입 반도체와 N타입 반도체를 붙이면 처음에는 자유 전자와 양공이 상대방 영역으로 넘어가서 상쇄된다. 그러다 보면 전자를 내주고 +가 된 도너 이온이나, 양공을 만들고 전자를 받아 -가 된 억셉터 이온들에 의한 전기장이 이 움직임을 방해한다.
최종적으로는 겉으로 보기에는 움직임이 멈추게 되는데, 이를 평형 상태라고 한다. \figurename~\ref{fig:eq pn junc}\을 보자.
\begin{figure}
   \centering
   \image{pn_junc.png}
   \caption{평형 상태에서의 PJ 접합}\label{fig:eq pn junc}
\end{figure}

\figurename~\ref{fig:eq pn junc}에서 동그라미는 움직일 수 있는 전자나 양공 캐리어, 네모는 움직일 수 없는 도너나 억셉터 이온을 의미한다.
양공 기준으로 움직임을 설명해보자. 전자는 반대 방향의 움직임으로 생각하면 되니까.
파란 화살표는 양공의 움직임(확산), 노란 화살표는 이온에 의한 전기력을 의미한다.
처음에는 농도차로 인해 양공은 오른쪽에서 왼쪽으로 이동한다.
그러다 보면 사라진 캐리어들로 인해 중화되지 않은 상태로 남은 이온들 간의 반대 방향 전기장이 점점 커지고, 이에 따라 양공은 왼쪽에서 오른쪽으로 드리프트하는 영향을 받는다.
결국에는 양공의 확산과 드리프트의 영향이 같아져서 겉으로 보기에는 양공의 움직임이 없는 것처럼 보이는 평형 상태에 도달한다. 전자 입장에서도 마찬가지이다.
이 상황을 수식으로 전개하면 다음과 같다.
\begin{IEEEeqnarray*}{rCl}
    -qD_p\frac{dp}{dx}&=&q\mu_ppE=q\mu_pp(-\frac{dV}{dx})\\
    \mu_p\frac{dV}{dx}&=&D_p\frac{1}{p}\frac{dp}{dx}\\
    \mu_p\int_{x_1}^{x_2}\frac{dV}{dx}dx&=&D_p\int_{x_1}^{x_2}\frac{1}{p}\frac{dp}{dx}dx\\
    \mu_p(V(x2)-V(x1))&=&D_p ln\frac{p(x_2)}{p(x_1)}\\
    V(x_2)-V(x_1)&=&\frac{D_p}{\mu_p}ln\frac{p(x_2)}{p(x_1)}
\end{IEEEeqnarray*}
두 지점 간의 전압 차이를 $V_{bi}$라 하자. 아인슈타인 관계식 $\frac{D_p}{\mu_p}=\frac{kT}{q}$을 적용하고, $x_1$에서의 양공 농도는 $p_n=\frac{{n_i}^2}{N_d}$, $x_2$에서의 양공 농도는 $p_p=N_A$임을 이용하면 $V_{bi}$는 다음과 같이 정리된다.
\begin{IEEEeqnarray*}{rCl}
    V_{bi}&=&\frac{kT}{q}ln\frac{p_p}{p_n}\\
    &=&\frac{kT}{q}ln\frac{N_A}{\frac{{n_i}^2}{N_d}}\\
    &=&\frac{kT}{q}ln\frac{N_AN_D}{{n_i}^2}\IEEEyesnumber
\end{IEEEeqnarray*}
이 $V_bi$를 내부 전위 장벽(Built In Potential)이라 한다. 에너지 밴드를 그림으로 그려보면 다음과 같다.
\begin{figure}[!hbp]
    \centering
    \image[height=5cm]{built_in_potential.png}
    \caption{평형 상태의 PN 접합 에너지 다이어그램}
\end{figure}
뒤에서 다루겠지만, 다이오드에 0.7 V를 건다고 하는데 그 전압의 정체가 바로 이 내부 전위 장벽의 크기이다. 0.7 V를 걸어서 이 전위 장벽을 눌러서 평평하게 만드는 것이다.
